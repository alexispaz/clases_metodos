\providecommand\xmin{-3}
\providecommand\xmax{3}
\providecommand\xstp{1}
\providecommand\xtra{0}
\providecommand\ymin{0}
\providecommand\ymax{3}
\providecommand\ystp{1}
\providecommand\nfra{1}

% Definir
\path(0,0) coordinate(O);
\path(\xmax,\ymax) coordinate(K1);          % Esquina del cuadrante 1
\path(K1-|{(\xmin,0)}) coordinate(K2); % Esquina del cuadrante 2
\path(K2|-{(0,\ymin)}) coordinate(K3); % Esquina del cuadrante 3

% Emergentes
\path(K3-|K1) coordinate(K4);       % Esquina del cuadrante 4
\path(O-|K1) coordinate(xp);       % Punta de eje +x
\path(O-|K2) coordinate(xn);       % Punta de eje -x
\path(O|-K1) coordinate(yp);       % Punta de eje +y
\path(O|-K3) coordinate(yn);       % Punta de eje -y


% grid
\draw[style=help lines, ystep=1, xstep=1] (K3) grid (K1);

% axes
\draw[->] (xn) -- (xp) node[right] {$x$};
% \draw[->] (yn) -- (yp) node[right] {$y$};

% xticks and yticks
% {\scriptsize
% \foreach \x in {-3,-2,...,3}  \path(O-|{(\x,0)}) node[below]{\x};
% % \foreach \y in {-1,0,...,3} \path(O|-{(0,\y)}) node[left]{\y};
% }

% generate and plot another a curve y = 0.1 x^2 + 2.5
% this generates the files figure.parabola.gnuplot and figure.parabola.table 
\draw[color=amarillo, domain=\xmin:\xmax] plot[smooth,id=parabola] function{sin(x-(\xtra))+2};
% \path[color=amarillo](0,3.5) node{$f(x)$};

% Diferencias finitas
\pgfmathsetmacro\xxmin{\xmin+(\xstp)}
\pgfmathsetmacro\xxmax{\xmax-(\xstp)}
\foreach \x in {\xmax,\xxmax,...,\xstp} {
  \pgfmathsetmacro\xi{int(\x)}
  \path(\x,0)  node[below]{\scriptsize $x_{n+\xi}$};
  \pgfmathsetmacro\y{sin((\x-(\xtra))/3.1415*180)+2}
  \draw(\x,0) -- (\x,\y) [fill]circle(.05) node[above=.2]{$f_{n+\xi}$};
}

\pgfmathsetmacro\y{sin((-(\xtra))/3.1415*180)+2}
\path(0,0) node[below]{\scriptsize $x_{n}$};
\draw(0,0) -- (0,\y) [fill]circle(.05) node[above=.2]{$f_{n}$};

\foreach \x in {\xmin,\xxmin,...,-\xstp} {
  \pgfmathsetmacro\xi{int(\x)}
  \path(\x,0) node[below]{\scriptsize $x_{n\xi}$};
  \pgfmathsetmacro\y{sin((\x-(\xtra))/3.1415*180)+2}
  \draw(\x,0) -- (\x,\y) [fill]circle(.05) node[above=.2]{$f_{n\xi}$};
}

% \path(-1,2.8) node[anchor=north west,align=left]{$x_r$=0};
    

