% Here is defined the layout and the text intrinsic properties

%% LAYOUT
\usepackage[
  a4paper,
  margin=3cm,
  %inner=2.8cm,
  %outer=2cm,
  %marginparwidth=2cm,
  %right=2.5cm,
  %marginparsep=-4mm,
  %headheight=13.6pt,
  %headheight=14.5pt,
  %headheight=15.28pt,
  %footnotesep=8mm,
  %showframe
]{geometry}

% SECTION FORMATING

\usepackage{titlesec}

% \titleformat{\chapter}[display]
% {\normalfont\huge\bfseries}{\chaptertitlename\ \thechapter}{20pt}{\Huge} % default book
% {\normalfont\bfseries\filcenter}{}{-2cm}{\LARGE}

% \titleformat{\section}
% {\normalfont\Large\bfseries}{\thesection}{1em}{} % default book
% {\normalfont\Large\bfseries}{\thesection}{1em}{}

% \titleformat{\subsection}
% {\normalfont\large\bfseries}{\thesubsection}{1em}{} % default book

% \titleformat{\subsubsection}
% {\normalfont\normalsize\bfseries}{\thesubsubsection}{1em}{} % default book

% \titleformat{\paragraph}[runin]
% {\normalfont\normalsize\bfseries}{\theparagraph}{1em}{} % default book

% \titleformat{\subparagraph}[runin]
% {\normalfont\normalsize\bfseries}{\thesubparagraph}{1em}{} % default book

% default book titlesec spacing for book class:
% \titlespacing*{\chapter} {0pt}{50pt}{40pt}
% \titlespacing*{\section} {0pt}{3.5ex plus 1ex minus .2ex}{2.3ex plus .2ex}
% \titlespacing*{\subsection} {0pt}{3.25ex plus 1ex minus .2ex}{1.5ex plus .2ex}
% \titlespacing*{\subsubsection}{0pt}{3.25ex plus 1ex minus .2ex}{1.5ex plus .2ex}
% \titlespacing*{\paragraph} {0pt}{3.25ex plus 1ex minus .2ex}{1em}
% \titlespacing*{\subparagraph} {\parindent}{3.25ex plus 1ex minus .2ex}{1em


  


             
% LINE SPACING

%If you want to adjust the spacing of text you should avoid to change the
%\baselinestretch as it changes the spacing for everything in the document,
%including footnotes and tables, which is usually not desirable (of course,
%if you want that effect then \renewcommand{\baselinestretch}{1.5} is
%correct)
%
%If you want a typographically pleasing result use
\usepackage{setspace}
%as this changes only the spacing of body text and the bibliography. As
%Paul Stanley noted, setting the document to \onehalfspace is less than a
%baselinestretch of 1.5. If you want the same effect use \setstretch{1.5}
%\onehalfspace
\setstretch{1.1}

%To get better control over the headers, one can use the package fancyhdr wich
%provides several commands that allows to customize the header and footer lines
%of your document.
\usepackage{fancyhdr}
% \renewcommand{\headrulewidth}{0.4pt}%
\renewcommand{\headrulewidth}{0pt} % remove horizontal line

\usepackage{lastpage}
\pagestyle{fancy} 


%PAGE BREAKS

%En libros de dos paginas es para que las paginas esten alineadas
%en el principio y el final .... pero puede generar espacios al medio.
%\raggedbottom
%opueesta a \flushbottom
%\renewcommand{\headrulewidth}{0.5pt}
 
%page breaks inside align are allowed, but avoided if possible
\allowdisplaybreaks[1]
%increase 1 to relax the hardness of the avoid.


%FOOTNOTE

% Footnotes with simbols, to avoid missunderstanid with powers
%\renewcommand*{\thefootnote}{\fnsymbol{footnote}}
\usepackage[perpage,symbol*]{footmisc}
\DefineFNsymbolsTM{myfnsymbols}{% def. from footmisc.sty "bringhurst" symbols
  \textdagger            \dagger  
  \textdaggerdbl         \ddagger 
  \textasteriskcentered  *        
  \textsection           \mathsection
  \textbardbl            \|%
  \textparagraph         \mathparagraph
}%
\setfnsymbol{myfnsymbols}
 
