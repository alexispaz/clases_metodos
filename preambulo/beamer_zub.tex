
% MODES
% article
% beamer
% trans
% handout: one page per frame (no pauses)
% presentation: not a mode, but can be used in <> commands to refer to beamer,
%               trans and handout modes.

% Zframe
% ======
% Un frame especial con coordenadas tikz estandarizadas
% Necesita llamarse despues de \RequirePackage{tikz,pgf}

\usepackage{xifthen}

% Fixing the seed
\pgfmathsetseed{\number\pdfrandomseed}

% Avoid some dimension too large (but take longer)
\usepackage{fp}
\usetikzlibrary{fixedpointarithmetic}
 

%\usepackage{setspace} % spacing environment

%% Prefix the name of the nodes in a scope
%% http://tex.stackexchange.com/a/128079/28411
%\makeatletter
%\tikzset{%
%  prefix node name/.code={%
%    \tikzset{name/.code={\edef\tikz@fig@name{#1 ##1}}}
%  }%
%}
%\makeatother
%The problem is that you can have to put the prefix when refers to internal
%nodes inside the scope

%\makeatletter
%\tikzset{privatebbox/.style={
%    %execute at begin scope={
%    %    % save the bounding box
%    %    \pgfpointanchor{current bounding box}{south west}
%    %    \pgfgetlastxy\tsx@outerbb@minx\tsx@outerbb@miny
%    %    \pgfpointanchor{current bounding box}{north east}
%    %    \pgfgetlastxy\tsx@outerbb@maxx\tsx@outerbb@maxy
%
%    %    % clear the bounding box
%    %    \pgfresetboundingbox
%    %},
%    %execute at end scope={
%    %    % do something useful with the scope bounding box
%    %    \draw (current bounding box.south west) rectangle (current bounding box.north east);
%
%    %    % reestablish the outer bounding box.
%    %    \path (\tsx@outerbb@minx, \tsx@outerbb@miny) rectangle (\tsx@outerbb@maxx,\tsx@outerbb@maxy);
%    %}
%}}
%\makeatother


% How to proyect into a plane in 3D
%http://tex.stackexchange.com/a/114266/28411



%tikz nots
\newcommand{\tkzn}[1]{\tikz[baseline=(#1.base),remember picture]\node(#1){};}
\newcommand{\tkznt}[2]{\tikz[baseline=(#1.base),remember picture]\node(#1){#2};}
\newcommand{\tkznot}[3]{\tikz[baseline=(#1.base),remember picture]\node(#1)[#2]{#3};}
\newcommand{\tkzpnot}[4]{\tikz[remember picture]\path(#1) node(#2)[#3]{#4};}
 

%use as {\paperwidth}{\paperheight}
\newlength{\bbw}
\newlength{\bbh}
\newcommand{\pcuad}[3][]{
  % First argument is an optional prefix for the coordinate names

  \setlength{\bbw}{#2}
  \setlength{\bbh}{#3}
  \useasboundingbox(0,0) rectangle (\bbw,\bbh);

  \path(.5\bbw,.5\bbh) coordinate(#1cp);
  \path(\bbw,0)        coordinate(#1se);
  \path(0,0)           coordinate(#1sw);
  \path(0,\bbh)        coordinate(#1nw);
  \path(\bbw,\bbh)     coordinate(#1ne);
 
  
  \path(\bbw,.5\bbh) coordinate(#1ep);
  \path(.5\bbw,0)    coordinate(#1sp);
  \path(.5\bbw,\bbh) coordinate(#1np);
  \path(0,.5\bbh)    coordinate(#1wp);
       
  \path(.75\bbw,.5\bbh) coordinate(#1hr);
  \path(.25\bbw,.5\bbh) coordinate(#1hl);
  \path(.5\bbw,.25\bbh) coordinate(#1vl);
  \path(.5\bbw,.75\bbh) coordinate(#1vu);

  \path(.75\bbw,.75\bbh) coordinate(#1c1);
  \path(.25\bbw,.75\bbh) coordinate(#1c2);
  \path(.25\bbw,.25\bbh) coordinate(#1c3);
  \path(.75\bbw,.25\bbh) coordinate(#1c4);


%  +-----------------------------------------------+
%  |nw                    np                     ne|
%  |                                               |
%  |                                               |
%  |          c2          vu          c1           |
%  |                                               |
%  |                                               |
%  |wp        hl          cp          hr         ep|
%  |                                               |
%  |                                               |
%  |          c3          vl          c4           |
%  |                                               |
%  |                                               |
%  |sw                    sp                     se|
%  +-----------------------------------------------+
}
 
        
\newcommand{\showcuad}[1][]{

  \draw[help lines,xstep=.5,ystep=.5,dashed,gray,opacity=0.2] (#1sw) grid (#1ne);
  \draw[help lines,xstep=1,ystep=1,gray,opacity=0.5]      (#1sw) grid (#1ne);

  \foreach \x in {0,2,...,10} {
      \node [anchor=north, gray,opacity=0.2,yshift=30] at (\x,0) {\tiny \bf \x};
      \node [anchor=east,gray,opacity=0.2,xshift=30] at (0,\x) {\tiny \bf  \x};
  }
             
  \fill[gray,opacity=0.2](#1se) circle(.1) node[anchor=south east]{#1se};
  \fill[gray,opacity=0.2](#1sw) circle(.1) node[anchor=south west]{#1sw};
  \fill[gray,opacity=0.2](#1ne) circle(.1) node[anchor=north east]{#1ne};
  \fill[gray,opacity=0.2](#1nw) circle(.1) node[anchor=north west]{#1nw};

  \fill[gray,opacity=0.2](#1hr) circle(.1) node[above]{#1hr};
  \fill[gray,opacity=0.2](#1hl) circle(.1) node[above]{#1hl};
  \fill[gray,opacity=0.2](#1vu) circle(.1) node[above]{#1vu};
  \fill[gray,opacity=0.2](#1vl) circle(.1) node[above]{#1vl};

  \fill[gray,opacity=0.2](#1sp) circle(.1) node[anchor=south]{#1sp};
  \fill[gray,opacity=0.2](#1wp) circle(.1) node[anchor=west] {#1wp};
  \fill[gray,opacity=0.2](#1np) circle(.1) node[anchor=north]{#1np};
  \fill[gray,opacity=0.2](#1ep) circle(.1) node[anchor=east] {#1ep};

  \fill[gray,opacity=0.2](#1cp) circle(.1) node[above]{#1cp};
  \fill[gray,opacity=0.2](#1c1) circle(.1) node[above]{#1c1};
  \fill[gray,opacity=0.2](#1c2) circle(.1) node[above]{#1c2};
  \fill[gray,opacity=0.2](#1c3) circle(.1) node[above]{#1c3}; 
  \fill[gray,opacity=0.2](#1c4) circle(.1) node[above]{#1c4};

  %\fill[red](cp|-c1) circle(.1) node[anchor=north]{cp|-c1};

}
 

%\usepackage{eso-pic} % This give absolut positioning in the page:
%\AddToShipoutPictureBG{}
%  All the picture commands which are parameters of an \AddToShipoutPictureBG
%  command will be added to the internal macro \ESO@HookIBG. This macro is part
%  of a zero-length picture environment with basepoint at the lower left corner of
%  the page (default) or at the upper left corner (package option ”texcoord”). The
%  picture environment will be shipped out at every new page.
%\AddToShipoutPictureBG* works like \AddToShipoutPictureBG but only for
%  the current page. It used the internal macro \ESO@HookIIBG which contents will
%  be removed automatically.

% The problem with this is that only, uncover and hidden of beamer don't work
%\newenvironment{frametikz}[1]{
%\RequirePackage{envrion}
%\NewEnviron{frametikz}{
%  \AddToShipoutPictureFG*\bgroup\put(-3,0)\bgroup \begin{tikzpicture} [overlay,#1] \pcuad
%  \BODY
%  \end{tikzpicture}\egroup\egroup
%}

% This is the chosen one for now
%\begin{frame}\par
%\frametitle<presentation>{Coalescence of Au and Co}
%\hspace*{-2pt}
%\vspace*{-2pt}
%\makebox[\linewidth][c]{
%\begin{tikzpicture}
%\pcuad{\paperwidth}{\paperheight}
%\path<1>(hr)++(1.15,0)  node{\includegraphics[width=0.25\textwidth]{./Imagenes/hddm/diag1.png}};
%\path<2>(hl)++(1.15,0)  node{\includegraphics[width=0.25\textwidth]{./Imagenes/hddm/diag1.png}};
%\end{tikzpicture}}
%\end{center} 
%\end{frame}
 

% It would be nice to have a library to mix tikz with the frames, and do the
% svg trick and also allow a subnode command to change the colors with beamer
% animations.

% Deprecated, see tikzmark
% Include a node for parts of an equation
%\newcommand\mytikzmark[3][]{%
%  \tikz[remember picture,baseline=(#2.base)]{\node(#2)[outer sep=0pt,#1]{#3};}%
%}

% Compatibilidad con babel 
% http://tex.stackexchange.com/a/166775/28411  

  % PGF/TikZ version 2.10
  % \tikzset{
  %   every picture/.append style={
  %     execute at begin picture={\deactivatequoting},
  %     execute at end picture={\activatequoting}
  %   }
  % }

  % PGF/TikZ version 3.0.0
  % \usetikzlibrary{babel}

% LIBRERIAS 
% ---------

% Para usar span
% \usetikzlibrary{positioning-plus}

% Zoom con una lupa. 
% NOTA: Hay que poner `spy using outline` en el tikzpicture
\usetikzlibrary{spy}

% Por alguna razon, spy cambia los colores foreground. Esto lo arregla.
% https://tex.stackexchange.com/a/204296/28411
\makeatletter
  \tikzset{%
    tikz@lib@reset@gs/.style={thin,solid,opaque,line cap=butt,line join=miter}
  }
\makeatother

% Decoración de nodos
\usetikzlibrary{
  decorations,
  decorations.fractals,
  decorations.shapes,
  decorations.text,
  decorations.pathmorphing,
  decorations.pathreplacing,
  decorations.footprints,
  decorations.markings,
}

%\usetikzlibrary{fpu} % Float point unit para operar
\usetikzlibrary{%
arrows,% Mas flechas que las de pgf
calc,% Para operar con coordnadas. Por ejemplo: ($(0,1)+(1,0)$) es (1,1) 
fit,%
patterns,%
plotmarks,%
shapes, % Reconoce la notacion .north .30 en nodos de distintas formas(circulo, elipse, etc)
% shapes.geometric,%
% shapes.misc,%
% shapes.symbols,%
% shapes.arrows,%
% shapes.callouts,%
% shapes.multipart,%
% shapes.gates.logic.US,%
% shapes.gates.logic.IEC,%
tikzmark, %the use of subnodes and tikzmarks, usefull to interconect tikz pictures
er,%?
automata,%?
backgrounds,% Manejo del background canvas del tikz ?
chains,%
topaths,%
trees,% Arboles, childs y esas cosas
petri,%
mindmap,% Arboles y childs con conexiones lindas, como neuronales
matrix,% Para funcionalidades en matrices como matrix of nodes
%calendar,%
folding,%
fadings,%
through,%
%positioning,%
scopes,%
hobby,%John Hobby’s algorithm, smooth curves as a list of cubic Bzier curves with endpoints at subsequent points in the list.
calligraphy,
% Al usar un text, tikz escribe caracter por caracter en un
%hbox. Entonces is se quiere imprimir uno indicandolo con más de uno (e.g.
%ó=\'{o}) es necesario encerrarlos entre braces.
shadows,% Sombras y transparencias
shadows.blur,
external% Externalización de gráficos
}

% LAYERS

% Creo una capa detras de la principal pero arriba del background
% es util para cosas como los diagramas de gantt, para
% que las flechas vayan por detras.
\pgfdeclarelayer{behind}
\pgfsetlayers{background,behind,main}

% To get \begin{scope}[zlevel=main] .....
\tikzset{zlevel/.style={%
        execute at begin scope={\pgfonlayer{#1}},
        execute at end scope={\endpgfonlayer}
}}

% to get \node[on layer=behind]....
%%% see https://tex.stackexchange.com/a/20426
\makeatletter
\pgfkeys{%
  /tikz/on layer/.code={
    \pgfonlayer{#1}\begingroup
    \aftergroup\endpgfonlayer
    \aftergroup\endgroup
  },
  /tikz/node on layer/.code={
    \gdef\node@@on@layer{%
      \setbox\tikz@tempbox=\hbox\bgroup\pgfonlayer{#1}\unhbox\tikz@tempbox\endpgfonlayer\egroup}
    \aftergroup\node@on@layer
  },
  /tikz/end node on layer/.code={
    \endpgfonlayer\endgroup\endgroup
  }
}
\def\node@on@layer{\aftergroup\node@@on@layer}
\makeatother
%%%
             


%                                                               Gantt charts
%------------------------------------------------------------------------------
% To perform Gantt Charts:
% A chart in which a series of horizontal lines shows the amount of work done or
% production completed in certain periods of time in relation to the amount
% planned for those periods.
\usepackage{pgfgantt}

% Esto es una forma de alterar el estilo de ganttbar
% \newganttchartelement{auxbar}{
%   bar/.style={
%   shape={chamfered rectangle},
%   chamfered rectangle corners={north east,south east},
%   inner sep=5pt,
%   draw=cyan!60!black,
%   very thick,
%   top color=white,
%   bottom color=cyan!50,
%   }
% }

\newganttchartelement{code}{
code/.style={
  shape={rectangle},
  inner sep=8pt,
  draw=pink!70!black,
  very thick,
  top color=white,
  bottom color=pink!50,
  },
}  
 
\newganttchartelement{sim}{
sim/.style={
  shape={chamfered rectangle},
  chamfered rectangle corners={north east,south east},
  inner sep=5pt,
  draw=cyan!60!black,
  very thick,
  top color=white,
  bottom color=cyan!50,
  },
  sim label font=\slshape,
  sim left shift=0,
  sim right shift=0,
}
                
\newganttchartelement{exp}{
exp/.style={
  shape={chamfered rectangle},
  chamfered rectangle corners={north east,south east},
  inner sep=5pt,
  draw=verde!60!black,
  very thick,
  top color=verde!50,
  bottom color=white,
  },
}  
               
\newganttchartelement{simexp}{
simexp/.style={
  shape={chamfered rectangle},
  chamfered rectangle corners={north east,south east},
  inner sep=7pt,
  draw=verde!70!cyan!60!black,
  ultra thick,
  top color=verde!50, 
  bottom color=cyan!50,
  },
}  
   

\newganttlinktype{mix}{
\ganttsetstartanchor{south}
\ganttsetendanchor{north}
\path(\xLeft, \yUpper)  ++(0,0.06)coordinate(b);
\path(\xRight, \yLower) ++(0,0.06) coordinate(e);
\path($(b)!.5!(e)$) coordinate(m);
\draw [on layer=behind,<-<,>=stealth,shorten >=-3,ultra thick, red] (b-|m) |- (m|-e) ;
}
 
\newganttlinktype{ston2}{
\ganttsetstartanchor{south}
\ganttsetendanchor{north}
\path(\xLeft, \yUpper)  ++(0,0.06)coordinate(b);
\path(\xRight, \yLower) ++(0,0.06) coordinate(e);
\path($(b)!.5!(e)$) coordinate(m);
\draw [on layer=behind,-<] (b-|m) |- (m|-e) ;
}
       

\newganttlinktype{stow}{
\ganttsetstartanchor{south}
\ganttsetendanchor{west}
  \draw[->,on layer=behind] (\xLeft, \yUpper) |- (\xRight, \yLower) ;
}
 
\newganttlinktype{eton}{
\ganttsetstartanchor{east}
\ganttsetendanchor{north}
  \draw[->,on layer=behind] (\xLeft, \yUpper) -| (\xRight, \yLower) ;
}
                   



         
%                                                               EXTERNALIZACION
%------------------------------------------------------------------------------

% Updated picture message
% \tikzset{external/verbose IO={false}}

%Esta linea es para que los graficos sean actualizados solo cuando sea necesario
%Esto se conoce como externalization graphics, y solo compila una vez los tikz
%picture y luego los importa del pdf adecuado en la carpeta Externos 
%\tikzsetexternalprefix{TikZ/} %%%%% NO, LO PONGO EN EL NAME DIRECTAMENTE

% Esto es para cambiar el nombre de cada imagen externalizada
  %\tikzsetnextfilename{AuPt300msErotEvib}

% This avoid to externalize the tikzpicture environment
\tikzset{/tikz/external/export={false}}

% Resolution of a tikzpicture that uses bitmap files
\def\tikzpngresolution{300 }
% dejar espacio al final!!!!
 
% Definition of the \includetikz.
% Use a separate file to include a long tikzpicture trough this command.
% It is better to have a folder with all this files togehter.
% A counter will increase in each call to this command. For portability, the
% important auxiliary files of the pgfplots and tikz externalization will be
% moved to this folder. Non important auxiliaries files will be deleted.
% The command admit two arguments. The first argmuent select one of the
% following behavior:
% *   this plot will be externalized and will not update any more (fast compilation).
% r   the picture will be updated using the stored input files readed again.
% R   the picture will be updated and all the input files readed again.
% H   hide this plot (fast compilation).
% h   this plot will be externalized and will not update any more (fast compilation).
% png use pdftoppm to include a bitmap of the plot (faster compilation for many data plots)
\def\includetikz[#1]#2{
  {
    % Activa la externalizacion
    \tikzset{external/export={true}} 
    \tikzsetnextfilename{#2}

    % Cuando usamos "raw gnuplot" en pgfplots el script y la tabla para gnuplot tambien
    %\pgfkeys{tikz/prefix={TikZ/}}
    %Contador para los caption
    %\setcounter{gnuplotcounter}{0}
%    \def\externalname{#2}
    

    % FORMA DE CARGA DEL TIKZPICTURE SEGUN ARGUMENTO #1
    \ifthenelse{\equal{#1}{\asterisco}}{
      % (*) NO LA FUERZA NI LA ESCONDE
      \input{#2.tex}
    }{
      \ifthenelse{\equal{#1}{H}}{
      % (H) SOLO LA ESCONDE.. NO PUEDO ESCONDER BOUNDING BOX
        \pgfkeys{/pgf/images/include external/.code={}}
        \input{#2.tex}
      }{
        % (h) LA ESCONDE Y LA FUERZA
        \ifthenelse{\equal{#1}{h}}{
          \tikzset{/tikz/external/mode={only graphics}}
          \input{#2.tex}
        }{
          % (R) REMARK, LA FUERZA Y BORRA LAS TABLAS O PNG SI LAS HUBIERA
          \ifthenelse{\equal{#1}{R}}{
            \tikzset{/tikz/external/force remake}
            \immediate\write18{rm -f #2*.table #2*.png}
            \input{#2.tex}
          }{
            % (r) REMARK, LA FUERZA
            \ifthenelse{\equal{#1}{r}}{
              \tikzset{/tikz/external/force remake}
              \input{#2.tex}
            }{
              % (png) GRAPHIC, UN INCLUDE DEL PDF CONVERTIDO A PNG LA
              % RESOLUCION ES \tikzpngresolution DPI. ESTO ES
              % EQUIVALENTE A * EN QUE NO REMARKEA
              \ifthenelse{\equal{#1}{png}}{
                \immediate\write18{if [ ! -f #20-1.png ]; then pdftoppm -png -r \tikzpngresolution #20.pdf #20; fi}

                \def\oldpdfimageresolution{\pdfimageresolution}
                \pdfimageresolution \tikzpngresolution
                \includegraphics{#20-1.png}
                \pdfimageresolution \oldpdfimageresolution 

                % La idea de esto es porque hay tikz que dan pdf que resulta lentisimo al
                % visualizar (Ejemplo pgfplots 3d graphics).  Porque vuelvo a 72 dpi???
                % porque parece que es la forma de mantener el tamaño en la conversion de
                % manera tal que en el documento final sea lo mismo meter un pdf o un
                % png. Es decir, cuando el latex lee un pdf le pone 72dpi de
                % resolucion... es algo mas complicado pero se vee en este fragmetno
                % googleado:
                
                %See pdftex manual:
                %  \pdfimageresolution (integer)
                %  The integer \pdfimageresolution parameter (unit: dots per inch, dpi) is
                %  a last resort value, used only for bitmap (jpeg, png, jbig2) images,
                %    but not for pdfs. The priorities are as follows: Often one image
                %      dimension (width or height) is stated explicitely in the TEX file. Then
                %      the image is properly scaled so that the aspect ratio is kept. If both
                %      image dimensions are given, the image will be stretched accordingly,
                %    whereby the aspect ratio might get distorted. Only if no image
                %      dimension is given in the TEX file, the image size will be calculated
                %      from its width and height in pixels, using the x and y resolution
                %      values normally contained in the image file. If one of these resolution
                %      values is missing or weird (either < 0 or > 65535), the
                %      \pdfimageresolution value will be used for both x and y > resolution,
                %    when calculating the image size. And if the \pdfimageresolution is
                %      zero, finally a default resolution of 72 dpi would be taken. The
                %      \pdfimageresolution is read when pdfTEX creates an image via
                %      \pdfximage. The given value is clipped to the range 0..65535 [dpi].
                %      Currently this parameter is used particularily for calculating the
                %      dimensions of jpeg images in exif format (unless at least one dimension
                %          is stated explicitely); the resolution values coming with exif files
                %      are currently ignored.
              }{
               }
             }
           }
         }
       }
     }
    %Imprimo el final del log por posibles errores, borro el log
    %\immediate\write18{cat #2#30.log | tail}
    %\immediate\write18{rm -f #2.log}


    \immediate\write18{rm -f #2*.vrb}
    %\immediate\write18{rm -f #2*.gnuplot}
    \immediate\write18{rm -f #2*.toc}
    \immediate\write18{rm -f #2*.nav}
    \immediate\write18{rm -f #2*.out}
    \immediate\write18{rm -f #2*.snm}
    \immediate\write18{rm -f #2*.dep}
    \immediate\write18{rm -f #2*.dpth} 
    %\immediate\write18{rm -f #2*.table} 
    %\immediate\write18{rm -f #2*.log} 


  } %To ensure group
}   


% DECORATION PATHS

% Style to hglight arrows with a white shadow
\tikzset{
    halo/.style={
        preaction={
            draw,
            white,
            line width=4pt,
            -
        },
        preaction={
            draw,
            white,
            ultra thick,
            shorten >=-2.5\pgflinewidth
        }
    }
}


% Sine function above a path (see http://tex.stackexchange.com/a/134516) this
% code gives some dimension too large
\tikzset{/pgf/decoration/.cd,
    number of sines/.initial=10,
    angle step/.initial=20,
}

\newdimen\tmpdimen
\pgfdeclaredecoration{complete sines}{initial}
{
    \state{initial}[
        width=+0pt,
        next state=move,
        persistent precomputation={
            \pgfmathparse{\pgfkeysvalueof{/pgf/decoration/angle step}}%
            \let\anglestep=\pgfmathresult%
            \let\currentangle=\pgfmathresult%
            \pgfmathsetlengthmacro{\pointsperanglestep}%
                {(\pgfdecoratedremainingdistance/\pgfkeysvalueof{/pgf/decoration/number of sines})/360*\anglestep}%
        }] {}
    \state{move}[width=+\pointsperanglestep, next state=draw]{
        \pgfpathmoveto{\pgfpointorigin}
    }
    \state{draw}[width=+\pointsperanglestep, switch if less than=1.25*\pointsperanglestep to final, % <- bit of a hack
        persistent postcomputation={
        \pgfmathparse{mod(\currentangle+\anglestep, 360)}%
        \let\currentangle=\pgfmathresult%
    }]{%
        \pgfmathsin{+\currentangle}%
        \tmpdimen=\pgfdecorationsegmentamplitude%
        \tmpdimen=\pgfmathresult\tmpdimen%
        \divide\tmpdimen by2\relax%
        \pgfpathlineto{\pgfqpoint{0pt}{\tmpdimen}}%
    }
    \state{final}{
        \ifdim\pgfdecoratedremainingdistance>0pt\relax
            \pgfpathlineto{\pgfpointdecoratedpathlast}
        \fi
   }
}



%ON BEAMER

% Daniel's code:
% http://tex.stackexchange.com/questions/55806/tikzpicture-in-beamer/55827#55827
  \tikzset{
    invisible/.style={opacity=0},
    visible on/.style={alt=#1{}{invisible}},
    alt/.code args={<#1>#2#3}{%
      \alt<#1>{\pgfkeysalso{#2}}{\pgfkeysalso{#3}} % \pgfkeysalso doesn't change the path
    },
  }



%% 3D coordinates rotation (http://tex.stackexchange.com/a/67588/28411)
% The matrix rotation is:
%$R=\begin{pmatrix}
%    \cos \alpha \cos \beta
%&   \textcolor{red}{\cos \alpha \sin \beta \sin \gamma - \sin \alpha \cos \gamma}
%&   \cos \alpha \sin \beta \cos \gamma + \sin \alpha \sin \gamma \\
%    \textcolor{red}{\sin \alpha \cos \beta}
%&   \sin \alpha \sin \beta \sin \gamma + \cos \alpha \cos \gamma
%&   \textcolor{red}{\sin \alpha \sin \beta \cos \gamma - \cos \alpha \sin \gamma} \\
%    - \sin \beta
%&   \textcolor{red}{\cos \beta \sin \gamma}
%&   \cos \beta \cos \gamma
%\end{pmatrix}\\p'=R\cdot p$

\newcommand{\savedx}{0}
\newcommand{\savedy}{0}
\newcommand{\savedz}{0}
 
\newcommand{\rotateRPY}[4][0/0/0]% point to be saved to \savedxyz, roll, pitch, yaw
{   \pgfmathsetmacro{\rollangle}{#2}
    \pgfmathsetmacro{\pitchangle}{#3}
    \pgfmathsetmacro{\yawangle}{#4}

    % to what vector is the x unit vector transformed, and which 2D vector is this?
    \pgfmathsetmacro{\newxx}{cos(\yawangle)*cos(\pitchangle)}% a
    \pgfmathsetmacro{\newxy}{sin(\yawangle)*cos(\pitchangle)}% d
    \pgfmathsetmacro{\newxz}{-sin(\pitchangle)}% g
    \path (\newxx,\newxy,\newxz);
    \pgfgetlastxy{\nxx}{\nxy};

    % to what vector is the y unit vector transformed, and which 2D vector is this?
    \pgfmathsetmacro{\newyx}{cos(\yawangle)*sin(\pitchangle)*sin(\rollangle)-sin(\yawangle)*cos(\rollangle)}% b
    \pgfmathsetmacro{\newyy}{sin(\yawangle)*sin(\pitchangle)*sin(\rollangle)+ cos(\yawangle)*cos(\rollangle)}% e
    \pgfmathsetmacro{\newyz}{cos(\pitchangle)*sin(\rollangle)}% h
    \path (\newyx,\newyy,\newyz);
    \pgfgetlastxy{\nyx}{\nyy};

    % to what vector is the z unit vector transformed, and which 2D vector is this?
    \pgfmathsetmacro{\newzx}{cos(\yawangle)*sin(\pitchangle)*cos(\rollangle)+ sin(\yawangle)*sin(\rollangle)}
    \pgfmathsetmacro{\newzy}{sin(\yawangle)*sin(\pitchangle)*cos(\rollangle)-cos(\yawangle)*sin(\rollangle)}
    \pgfmathsetmacro{\newzz}{cos(\pitchangle)*cos(\rollangle)}
    \path (\newzx,\newzy,\newzz);
    \pgfgetlastxy{\nzx}{\nzy};

    % transform the point given by #1
    \foreach \x/\y/\z in {#1}
    {   \pgfmathsetmacro{\transformedx}{\x*\newxx+\y*\newyx+\z*\newzx}
        \pgfmathsetmacro{\transformedy}{\x*\newxy+\y*\newyy+\z*\newzy}
        \pgfmathsetmacro{\transformedz}{\x*\newxz+\y*\newyz+\z*\newzz}
        \xdef\savedx{\transformedx}
        \xdef\savedy{\transformedy}
        \xdef\savedz{\transformedz}     
    }
}

\tikzset{RPY/.style={x={(\nxx,\nxy)},y={(\nyx,\nyy)},z={(\nzx,\nzy)}}}
\tikzset{RPYxzy/.style={x={(\nxx,\nxy)},z={(\nyx,\nyy)},y={(\nzx,\nzy)}}}





% HANDRAW

% http://tex.stackexchange.com/a/218483/28411
\pgfdeclaredecoration{penciline}{initial}{
  \state{initial}[width=+\pgfdecoratedinputsegmentremainingdistance,
    auto corner on length=1pt,
  ]{
    \ifthenelse
        {\pgfkeysvalueof{/tikz/penciline/jag ratio}=0} {
          \pgfpathcurveto%
              {% 1st control point
                \pgfpoint
                    {\pgfdecoratedinputsegmentremainingdistance/2}
                    {2*rnd*\pgfdecorationsegmentamplitude}
              }
              {%% 2nd control point
                \pgfpoint
                %% Make sure random number is always between origin and target points
                    {\pgfdecoratedinputsegmentremainingdistance/2}
                    {2*rnd*\pgfdecorationsegmentamplitude}
              }
              {% 2nd point (1st one is implicit)
                \pgfpointadd
                    {\pgfpointdecoratedinputsegmentlast}
                    {\pgfpoint{0*rand*1pt}{0*rand*1pt}}
              }          
        } {
          \pgfpathcurveto%
              {% 1st control point
                \pgfpoint
                    {\pgfdecoratedinputsegmentremainingdistance*rnd*1pt}
                    {\pgfkeysvalueof{/tikz/penciline/jag ratio}*
                      rand*\pgfdecorationsegmentamplitude}
              }
              {%% 2nd control point
                \pgfpoint
                %% Make sure random number is always between origin and target points
                    {(.5+0.25*rand)*\pgfdecoratedinputsegmentremainingdistance}
                    {\pgfkeysvalueof{/tikz/penciline/jag ratio}*
                      rand*\pgfdecorationsegmentamplitude}
              }
              {% 2nd point (1st one is implicit)
                \pgfpointadd
                    {\pgfpointdecoratedinputsegmentlast}
                    {\pgfpoint{rand*1pt}{rand*1pt}}
              }
        }
  }
  \state{final}{}
}

\tikzset{
  penciline/.code={\pgfqkeys{/tikz/penciline}{#1}},
  penciline={
    jag ratio/.initial=5,
    decoration/.initial = penciline,
  },
  penciline/.style = {
    decorate,
    %%decoration={\pgfkeysvalueof{/tikz/penciline/decoration}},
    penciline/.cd,
    #1,
    /tikz/.cd,
  },
  decorate,
  decoration={\pgfkeysvalueof{/tikz/penciline/decoration}},
}

%Use as \draw[penciline={jag ratio=6}]...


% The best is 
%http://tex.stackexchange.com/a/39299/28411
%pencildraw/.style={
%  black!75,
%  decorate,
%  decoration={random steps,segment length=2pt,amplitude=0.5pt}
%  } 

%%% http://tex.stackexchange.com/questions/39296/simulating-hand-drawn-lines: Alain Matthes
%\pgfdeclaredecoration{free hand}{start}
%{
%  \state{start}[width = +0pt,
%                next state=step,
%                persistent precomputation = \pgfdecoratepathhascornerstrue]{}
%  \state{step}[auto end on length    = 3pt,
%               auto corner on length = 3pt,               
%               width=+2pt]
%  {
%    \pgfpathlineto{
%      \pgfpointadd
%      {\pgfpoint{2pt}{0pt}}
%      {\pgfpoint{rand*0.15pt}{rand*0.15pt}}
%    }
%  }
%  \state{final}
%  {}
%}
% \tikzset{free hand/.style={
%    decorate,
%    decoration={free hand}
%    }
% } 
%\def\freedraw#1;{\draw[free hand] #1;}

% Line ends or Arrow styles

\tikzset{cuota/.style={
  postaction={decorate,
    decoration={markings,
      mark=at position 0 with  {
            \begin{scope}[xslant=0.2]
            \draw[line width=1pt,-] (0pt,-3pt) -- (0pt,3pt);
            \end{scope}
      },
      mark=at position 1 with  {
            \begin{scope}[xslant=0.2]
            \draw[line width=1pt,-] (0pt,-3pt) -- (0pt,3pt);
            \end{scope}
      }
    }
  }
}}
   




% To put a frame to an equation
\newcommand{\tikzframeab}[4]{%
  \tikz[remember picture,overlay] \path[#1]([shift={(-#2,#3)}] pic cs:a) rectangle ([shift={(#2,-#4)}] pic cs:b);
}
% Use:
% \begin{equation}
% \tikzmark{a} 2+2 \tikzmark{b}
% \end{equation}
% tikzframeab{draw}{2ex}{2.5ex}{1.5ex}
                                





% \usepackage[rows=10,cols=10]{beamersozi}
\tikzset{% 
  rec/.style={rectangle, draw=bg, rounded corners=1mm,fill=white,inner sep=3},
  recorte/.style={draw=bg,fill=white,draw, minimum height=2cm, minimum width=3cm,
         decorate, decoration={random steps,segment length=3,amplitude=2}},
}
 

\pgfdeclareradialshading{myshading}{\pgfpointorigin}{%
color(0mm)=(pgftransparent!0);%
color(5mm)=(pgftransparent!10);%
color(8mm)=(pgftransparent!50);%
color(15mm)=(pgftransparent!100)%
}
\pgfdeclarefading{myfading}{\pgfuseshading{myshading}}  
\tikzset{ partial ellipse/.style args={#1:#2:#3}{insert path={+ (#1:#3) arc (#1:#2:#3)}}}


\pgfdeclareradialshading{myshading2}{\pgfpointorigin}{%
color(0mm)=(pgftransparent!0);%
color(1mm)=(pgftransparent!50);%
color(5mm)=(pgftransparent!70);%
color(8mm)=(pgftransparent!80);%
color(15mm)=(pgftransparent!90);%
color(20mm)=(pgftransparent!100)%
}
\pgfdeclarefading{myfading2}{\pgfuseshading{myshading2}}  
\tikzset{ partial ellipse/.style args={#1:#2:#3}{insert path={+ (#1:#3) arc (#1:#2:#3)}}}

 
% Estilos
% =======
% Estos estilos son mas portables si se pegan en cada tikzpicture.

% To avoid pading around figures in nodes with figures
\tikzset{graphics/.style={inner sep=0,outer sep=0}}

% To scale even linewidth and fonts
%\tikzset{scaleall/.style={scale=#1, every node/.append style={scale=#1}}}
\tikzset{scaleall/.style={scale=#1, transform shape}}
  
% Control `blur shadow`
% every shadow/.style={shadow xshift=-2ex, shadow yshift=-2ex}, 

% El cuadro redondeado
% cuadro/.style={rectangle, rounded corners=1mm,fill=white,inner sep=3,text opacity=1},

% El recorte con tijera random 
% recorte/.style={draw=azul,fill=white,thick,draw, minimum height=2cm, minimum width=3cm,
%        decorate, decoration={random steps,segment length=3,amplitude=2}},

% Flecha filosa
% flecha/.style={->,>=stealth,verde,line width=1mm},

% Teorema, definiciones, algoritmos
\tikzset{
 cuadro/.style={execute at begin node=\setlength{\baselineskip}{4ex},rectangle, rounded corners=1ex,fill opacity=0.2,text opacity=1, anchor=north west,inner sep=6,align=left},
 def/.style={cuadro,draw=amarillo,fill=verde},
 defhead/.style={def,inner sep=4,fill opacity=1,anchor=east,font={\bfseries}},
 proc/.style={cuadro,draw=verde,fill=verde},
 prochead/.style={proc,inner sep=4,fill opacity=1,anchor=east,font={\bfseries}},
 theo/.style={cuadro,draw=naranja,fill=naranja},
 theohead/.style={theo,inner sep=4,fill opacity=1,anchor=east,font={\bfseries}},
 alg/.style={cuadro,draw=celeste,fill=celeste},
 alghead/.style={alg,inner sep=4,fill opacity=1,anchor=east,font={\bfseries}},
}

\tikzset{cc/.style={anchor=north, inner sep=1, scale=0.8}}
\newcommand\by[2]{\tiny\color{gray}\href{#2}{#1}}

% Otros del monton  
% tit/.style={font=\Large\color{bg},anchor=north west,text width=\textwidth},
% nube/.style={cloud, cloud puffs=10.8,cloud puff arc=110, aspect=2, fill=white
%  ,double,double distance=1mm, inner sep=0mm,align=center,
%  font=\small\color{bg}}
% itema/.style={anchor=north west},
% flecha/.style={ >=latex, ->, shorten <={-1.5} },
% segmento/.style={ ultra thick,
% {Circle[width=4,length=4]}-{Circle[width=4,length=4]}, shorten <={-2},
% shorten >={-2} }, 
% semirecta/.style={ {Circle[width=3,length=3]}-latex, shorten <={-1.5} },
% def/.style={ anchor=north west, align=justify, text=fg, rectangle, rounded
% corners=1ex, thick, draw=verde, % font={\baselineskip=3.6mm}, },
% txt/.style={ anchor=north west, align=justify, text=fg, rectangle, }  
  


% Evironment that read its content before processing
\usepackage{environ}

% No space on left
\setbeamersize{text margin left=0cm}

% Avoid ignorenonframetext
\NewEnviron{zframe}[2][<1->]{%
\begin{frame}#1[plain]
\providecommand\CC{\path(se) node[anchor=south east]{\small\color{gray} S. A. Paz CC BY 4.0};}
\begin{tikzpicture}[#2]% yscale=0.9, % with frametitle

% Set relevant coordinates for the frame  
\pcuad{\paperwidth}{\paperheight}
\path(1,8.5) coordinate(t);


% Include name/licence mark
\CC  

\BODY

% Show relevant coordinates
% \showcuad
% \fill[gray,opacity=0.3](t) circle(.1) node[anchor=south east]{t};

\end{tikzpicture}
\end{frame}
}

% Peliculas
% =========

% \usepackage{movie15} % obsolete, superseed by media9 but not in gif.. puaj
% \usepackage{media9} % use the Flash Player plugin.. puaj
% \usepackage{animate} % uses a QuickTime sub-plugin.. puaj
\usepackage{multimedia} % avi

% Animaciones
% -----------

% Beamer tiene una forma de animar intrinseca.
% Lastima que esto solo funciona con Acrobat Reader.. puaj
%
% \animate<2-9>
% %\animatevalue<1-9>{\x}{0}{0.5} % Only for counters and dimensions
% \foreach \x in {0.1,0.2,...,1.0} {\only{
%   \path<+>(vu) ++(0,0) -- ++(0,3) node[pos=\x]{$\color{mcyellow}\text{bath at }T$};
% }}
 


%------------------------------------------------------------------------------
%                             Bibliography
%------------------------------------------------------------------------------


  \setbeamertemplate{bibliography item}[triangle]
  \setbeamercolor{bibliography entry author}{fg=white}
  \setbeamercolor{bibliography entry title}{fg=white}
  \setbeamercolor{bibliography entry journal}{fg=white}
  \setbeamercolor{bibliography entry note}{fg=white}
  \setbeamerfont{bibliography entry author}{size=\small}
%  \setbeamertemplate{bibliography item}[triangle]
%  \setbeamercolor{bibliography entry author}{fg=white}
%  \setbeamerfont{bibliography entry author}{size=\small}
%  {\tiny \color{black}\cite{Foiles011986} \cite{Johnson061989}}
%  \bibliographystyle{Bibliografia/elsart-num.bst}
%  \bibliography{alexispaz}

  \setbeamertemplate{bibliography item}[default] %  little article icon as the reference


% Margin size
% -----------
% \setbeamersize{sidebar width left=2cm}
%  • sidebar width left= TEX dimension sets the size of the left sidebar. Currently, this command
%    should be given before a shading is installed for the sidebar canvas.
%  • sidebar width right= TEX dimension sets the size of the right sidebar.
%  • description width= TEX dimension sets the default width of description labels, see Section 11.1.
%  • description width of= text sets the default width of description labels to the width of the
%      text , see Section 11.1.
%  • mini frame size= TEX dimension sets the size of mini frames in a navigation bar. When two
%    mini frame icons are shown alongside each other, their left end points are TEX dimension far
%    apart.
%  • mini frame offset= TEX dimension set an additional vertical offset that is added to the mini
%    frame size when arranging mini frames vertically.


%http://tex.stackexchange.com/a/75651/28411
%The default page size in a beamer presentation is 12.8cm x 9.6cm (a 4:3 aspect
%ratio) with a font size of 11pt. The page size is set using geometry using the
%following additional default settings:
%
%\RequirePackage[%
%  papersize={\beamer@paperwidth,\beamer@paperheight},
%  hmargin=1cm,%
%  vmargin=0cm,%
%  head=0.5cm,% might be changed later
%  headsep=0pt,%
%  foot=0.5cm% might be changed later
%  ]{geometry}% http://ctan.org/pkg/geometry
%In the above, \beamer@paperwidth and \beamer@paperheight is set based on the
%required aspect ratio. Also as mentioned in comments above, the values of head
%and foot may change, depending on the requirements provided by the
%template/theme used.   
%\DeclareOptionBeamer{aspectratio}[43]{%
%  \ifnum#1=1610% 16:10
%    \beamer@paperwidth 16.00cm%
%    \beamer@paperheight 10.00cm%
%  \else\ifnum#1=169% 16:9
%    \beamer@paperwidth 16.00cm%
%    \beamer@paperheight 9.00cm%
%  \else\ifnum#1=149% 14:9
%    \beamer@paperwidth 14.00cm%
%    \beamer@paperheight 9.00cm%
%  \else\ifnum#1=54% 5:4
%    \beamer@paperwidth 12.50cm%
%    \beamer@paperheight 10.00cm%
%  \else\ifnum#1=43% 4:3
%    \beamer@paperwidth 12.80cm%
%    \beamer@paperheight 9.60cm%
%  \else\ifnum#1=32% 3:2
%    \beamer@paperwidth 13.50cm%
%    \beamer@paperheight 9.00cm%
%  \fi\fi\fi\fi\fi\fi%
%}
                      




% Background 
% ----------

%Para cambiar la imagen de fondo, redefino el siguiente comando..
%   usar asi:
%   \framebg{Imagenes/blancosygrises.png} 
\newcommand{\framebg}[1]{\setbeamertemplate{background canvas}{\includegraphics[width=\paperwidth,height=\paperheight]{#1}}}

% \setbeamertemplate{background canvas} % Beamer-Template/-Color/-Font 
% \setbeamertemplate{background canvas}[default] 
% \setbeamertemplate{background canvas}[vertical shading][ color options ] installs a vertically shaded background. 
%     – top= color specifies the color at the top of the page. By default, 25% of the foreground of
%       the beamer-color palette primary is used.
%     – bottom= color specifies the color at the bottom of the page. By default, the background of
%       normal text at the moment of invocation of this command is used.
%     – middle= color specifies the color for the middle of the page. Thus, if this option is given, the
%       shading changes from the bottom color to this color and then to the top color.
%     – midpoint= factor specifies at which point of the page the middle color is used. A factor of 0
%       is the bottom of the page, a factor of 1 is the top. The default, which is 0.5 is in the middle.
% \setbeamertemplate{background} % Beamer-Template/-Color/-Font 
% \setbeamertemplate{background}[grid][step=1cm] % places a grid on the background. 
%     – step= dimension specifies the distance between grid lines. The default is 0.5cm.
%     – color= color specifies the color of the grid lines. The default is 10% foreground. 
 

% Frames
% ======

% Title
% -----

% \setbeamertemplate{frametitle} % Beamer-Template/-Color/-Font 
% \setbeamertemplate{frametitle}[default][left] % left, center, right 
% \setbeamertemplate{frametitle}[shadow theme]
% \setbeamertemplate{frametitle}[sidebar theme] 
% \setbeamertemplate{frametitle}[smoothbars theme] 
% \setbeamertemplate{frametitle}[smoothtree theme] 
 
 
\setbeamercolor*{frametitle}{fg=black,bg=mcgreen}
% Elemente deren Farbe veraendert werden kann

\setbeamercolor{normal text}{fg=white,bg=black} % (NO TOCAR)
% Cunado se establece esto, el beamer cambia el color
% implicito por default de las cosas. 

\defbeamertemplate<article>{frame begin}{lined}{\par\noindent\rule{\textwidth}{1pt}\par}
\defbeamertemplate<article>{frame end}{lined}{\par\noindent\rule{\textwidth}{1pt}\par}

\newcounter{framebox}
\defbeamertemplate<article>{frame begin}{tikzed}{\par\noindent\stepcounter{framebox}\tikz[remember picture,overlay] \path (-1ex,0) coordinate (frame top \the\value{framebox});}
\defbeamertemplate<article>{frame end}{tikzed}{\hspace*{\fill}\tikz[remember picture,overlay] \draw (frame top \the\value{framebox}) rectangle (1ex,0);\par}

\defbeamertemplate<article>{frame begin}{centered}{\par\noindent\begin{center}\par}
\defbeamertemplate<article>{frame end}{centered}{\par\noindent\end{center}\par}

\defbeamertemplate<article>{frame begin}{minipage}{\par\noindent\begin{minipage}{0.8\textwidth}\par}
\defbeamertemplate<article>{frame end}{minipage}{\par\noindent\end{minipage}\par}

\setbeamertemplate{frame begin}[centered]
\setbeamertemplate{frame end}[centered]
 
% \setbeamercolor*{example text}
% \setbeamercolor*{titlelike}


% \setbeamercolor*{math text}
% \setbeamercolor*{math text inlined}
% \setbeamercolor*{math text displayed}
% \setbeamercolor*{normal text in math text}

% \usebeamercolor[fg]{normal text}
% \setbeamercolor{normal text}{fg=black,bg=mylightgrey}

% Palette:
%\setbeamercolor{palette primary}{fg=red,bg=green}            
%\setbeamercolor{palette secondary}{fg=orange,bg=verdecmyk!80!black} 
%\setbeamercolor{palette tertiary}{fg=gray,bg=violet}          
%\setbeamercolor{palette quaternary}{fg=lily,bg=red}        
 

% Sidebar
% -------
%\useoutertheme[%
%  left, % sidebar links
%% right, % sidebar rechts
%  height=25pt, % offset headline
%  width=45pt, % sidebar rechts
%% hideallsubsections, % nur sections werden angezeigt
%% hideothersubsections, % nur subsections der aktuellen section werden angezeigt
%]{sidebar}
 

%\setbeamercolor*{sidebar}{fg=black,bg=verdecmyk}
%\setbeamercolor*{author in sidebar}{fg=black}
%\setbeamercolor*{title in sidebar}{fg=black}
%\setbeamercolor*{section/subsection in sidebar}{fg=black,bg=blanco}

% \setbeamertemplate{sidebar} % Beamer-Template/-Color/-Font (Parent)
% \setbeamertemplate{sidebar left} % Beamer-Template/-Color/-Font 
% \setbeamertemplate{sidebar right} % Beamer-Template/-Color/-Font 
% \setbeamertemplate{sidebar canvas left} % Beamer-Template 
% \setbeamertemplate{sidebar canvas right} % Beamer-Template 



% \setbeamercolor{palette sidebar primary}{fg=verdeposter!50!verdecmyk,bg=black!70!verdecmyk}
% \setbeamercolor{palette sidebar secondary}{fg=verdeposter,bg=black!90!verdecmyk}
%\setbeamercolor{palette sidebar tertiary}{fg=black,bg=red}
%\setbeamercolor{palette sidebar quaternary}{fg=black,bg=red} 


\setbeamerfont*{sidebar}{size=\tiny}


% Head/foot
% ---------

\setbeamercolor*{author in head/foot}{fg=yellow,bg=black}
\setbeamercolor*{title in head/foot}{fg=black,bg=yellow}
%\setbeamercolor*{date in head/foot}{parent=palette primary}
%\setbeamercolor*{section in head/foot}{parent=palette secondary} %tertiary
%\setbeamercolor*{subsection in head/foot}{parent=palette primary} 
%\setbeamercolor*{separation line}{fg=red,bg=red,red}
%\setbeamercolor*{upper separation line head}{fg=red,bg=red,red}
%\setbeamercolor*{separation line}{fg=red,bg=red,red}
%\setbeamercolor*{middle separation line head}{fg=red,bg=red,red}
%\setbeamercolor*{separation line}{fg=red,bg=red,red}
%\setbeamercolor*{lower separation line head}{fg=red,bg=red,red}
%\setbeamercolor*{upper separation line foot}{fg=red,bg=red,red}
%\setbeamercolor*{middle separation line foot}{fg=red,bg=red,red}
%\setbeamercolor*{lower separation line foot} {fg=red,bg=red,red}


% \useoutertheme{default}

% \useoutertheme{infolines}


% \useoutertheme[
% % 	hooks, % Einruecken der Abschnittsueberschriften in der Kopfzeile
% ]{tree}

% wie tree, ohne die Linien
% \useoutertheme{smoothtree}



 % \useoutertheme[%
% 	footline=empty, % suppressed the footline (default).
% % 	footline=authorinstitute, %shows the author's name and the institute in the footline.
% % 	footline=authortitle, % shows the author's name and the title in the footline.
% % 	footline=institutetitle, % shows the institute and the title in the footline.
% % 	footline=authorinstitutetitle, % shows the author's name, the institute, and the title in the footline.
% ]{miniframes}

% % This theme installs a headline in which, on the left, the sections of the talk are shown and, on the right,
% % the subsections of the current section. If the class option compress has been given, the sections and
% % subsections will be put in one line; normally there is one line per section or subsection.
% \useoutertheme{split}
% The colors are taken from palette primary and palette fourth.
 
% \useoutertheme[%
% % 	subsection= true,  % or false shows or suppresses line showing the subsection in the headline.
% ]{smoothbars}

% % This layout theme extends the split theme by putting a horizontal shading behind the frame title and
% % adding a little 'shadow' at the bottom of the headline.
% \useoutertheme{shadow}

% \setbeamertemplate{headline} % Beamer-Template/-Color/-Font 
% \setbeamertemplate{headline}
% {%
%   \begin{beamercolorbox}{section in head/foot}
%     \vskip2pt\insertnavigation{\paperwidth}\vskip2pt
%   \end{beamercolorbox}%
% }
% \setbeamertemplate{headline}[default] % The default is just an empty headline. 
% \setbeamertemplate{headline}[infolines theme] 
% \setbeamertemplate{headline}[miniframes theme] 
% \setbeamertemplate{headline}[sidebar theme] 
% \setbeamertemplate{headline}[smoothtree theme]
% \setbeamertemplate{headline}[smoothbars theme]
% \setbeamertemplate{headline}[tree] 
% \setbeamertemplate{headline}[split theme] 
% \setbeamertemplate{headline}[text line]{ text } % The headline is typeset with 'text' 

% \setbeamertemplate{footline} % Beamer-Template/-Color/-Font 
% \setbeamertemplate{footline}[default] 
% \setbeamertemplate{footline}[infolines theme] 
% \setbeamertemplate{footline}[miniframes theme] 
% \setbeamertemplate{footline}[page number] 
% \setbeamertemplate{footline}[frame number] 
% \setbeamertemplate{footline}[split] 
% \setbeamertemplate{footline}[text line]{ text } 

% \setbeamertemplate{page number in head/foot} % Beamer-Color/-Font 
 


%------------------------------------------------------------------------------
%                             NAVIGATION
%------------------------------------------------------------------------------ 


%                                                                    HYPERLINKS
%------------------------------------------------------------------------------
%%% 10.1 Adding Hyperlinks and Buttons
% \setbeamertemplate{button} % Beamer-Template/-Color/-Font
% • \insertbuttontext inserts the text of the current button. Inside “Goto-Buttons” (see below)
%   this text is prefixed by the insert \insertgotosymbol and similarly for skip and return buttons.
% • \insertgotosymbol This text is inserted at the beginning of goto buttons. Redefine this
%   command to change the symbol.
%   Example: \renewcommand{\insertgotosymbol}{\somearrowcommand}
% • \insertskipsymbol This text is inserted at the beginning of skip buttons.
% • \insertreturnsymbol This text is inserted at the beginning of return buttons. 





%BAR (funktioniert nur mit miniframe Themes)

% \setbeamertemplate{mini frames}[default] % shows small circles as mini frames.
\setbeamertemplate{mini frames}[box] % shows small rectangles as mini frames.
% \setbeamertemplate{mini frames}[tick] % shows small vertical bars as mini frames.

% \setbeamertemplate{mini frame} % Beamer-Template/-Color/-Font 
% \setbeamertemplate{mini frame in current subsection} % Beamer-Template 
% \setbeamertemplate{mini frame in other subsection} % Beamer-Template 
% \setbeamertemplate{mini frame in other subsection}[default][20]

% \setbeamertemplate{section in head/foot} % Beamer-Template/-Color/-Font 
% \setbeamertemplate{section in head/foot shaded} % Beamer-Template 
% \setbeamertemplate{section in head/foot shaded}[default][20]
% \setbeamertemplate{section in sidebar} % Beamer-Template/-Color/-Font 
% \setbeamertemplate{section in sidebar}[sidebar theme]
% \setbeamertemplate{subsection in head/foot} % Beamer-Template/-Color/-Font 
% \setbeamertemplate{subsection in head/foot shaded} % Beamer-Template 
% \setbeamertemplate{section in head/foot shaded}[default][20]
% \setbeamertemplate{subsection in head/foot shaded}[default][20]
% \setbeamertemplate{subsection in sidebar} % Beamer-Template/-Color/-Font 
% \setbeamertemplate{subsection in sidebar shaded} % Beamer-Template 
% \setbeamertemplate{subsubsection in head/foot} % Beamer-Template/-Color/-Font 
% \setbeamertemplate{subsubsection in head/foot shaded} % Beamer-Template 
% \setbeamertemplate{subsubsection in head/foot shaded}[default][20] 
% \setbeamertemplate{subsubsection in sidebar} % Beamer-Template/-Color/-Font
% \setbeamertemplate{subsubsection in sidebar shaded} % Beamer-Template 

%SYMBOLS
%%% Beamer-Template/-Color/-Font navigation symbols
\setbeamertemplate{navigation symbols}{} % suppresses all navigation symbols:
% \setbeamertemplate{navigation symbols}[horizontal] % Organizes the navigation symbols horizontally.
% \setbeamertemplate{navigation symbols}[vertical] % Organizes the navigation symbols vertically.
% \setbeamertemplate{navigation symbols}[only frame symbol] % Shows only the navigational symbol for navigating frames.

%THE LOGO	
% \setbeamertemplate{logo} % Beamer-Template/-Color/-Font 




%------------------------------------------------------------------------------
%                           ITEMIZE, ENUMERATE, BOXES
%------------------------------------------------------------------------------

% \setbeamertemplate{items} % parent template of itemize items and enumerate items
% \setbeamertemplate{itemize items} % Parent Beamer-Template 
%\setbeamertemplate{itemize items}[triangle]
% \setbeamertemplate{itemize items}[circle] 
% \setbeamertemplate{itemize items}[square] 
%  \setbeamertemplate{itemize items}[ball] 
%  \setbeamercolor{item}{fg=red}
% \setbeamertemplate{itemize item} % Beamer-Template/-Color/-Font 
% \setbeamertemplate{itemize subitem} % Beamer-Template/-Color/-Font 
% \setbeamertemplate{itemize subsubitem} % Beamer-Template/-Color/-Font 
% -------------------------------
% \setbeamertemplate{enumerate items}[default] % Numbered 
% \setbeamertemplate{enumerate items}[circle] % Places the numbers inside little circles. 
% \setbeamertemplate{enumerate items}[square] % Places the numbers on little squares.
%  \setbeamertemplate{enumerate items}[ball] % “Projects” the numbers onto little balls.
% -------------------------------
% \setbeamertemplate{enumerate items} % Parent Beamer-Template 
% \setbeamertemplate{enumerate item} % Beamer-Template/-Color/-Font 
%  • \insertenumlabel inserts the current number of the top-level enumeration (as an Arabic
% 	   number). This insert is also available in the next two templates.
% \setbeamertemplate{enumerate subitem} % Beamer-Template/-Color/-Font 
% \setbeamertemplate{enumerate subitem}{\insertenumlabel-\insertsubenumlabel}
%  • \insertsubenumlabel inserts the current number of the second-level enumeration (as an Ara-
%   	bic number).
% \setbeamertemplate{enumerate subsubitem} % Beamer-Template/-Color/-Font 
%  • \insertsubsubenumlabel inserts the current number of the second-level enumeration (as an
%     Arabic number).
% \setbeamertemplate{enumerate mini template} % Beamer-Template/-Color/-Font 
%  • \insertenumlabel inserts the current number rendered by this mini template. For example,
%     if the mini template is (i) and this command is used in the fourth item, \insertenumlabel
%     would yield (iv).
% \setbeamertemplate{itemize/enumerate body begin} % Beamer-Template 
% \setbeamertemplate{itemize/enumerate body end} % Beamer-Template 
% -------------------------------
% \setbeamertemplate{description item}[default] % Beamer-Template/-Color/-Font 
%      • \insertdescriptionitem inserts the text of the current description item.
% -------------------------------
% \setbeamertemplate{item}  % Beamer-Color/-Font 
% \setbeamertemplate{item projected}   % Beamer-Color/-Font 
% \setbeamertemplate{subitem}  % Beamer-Color/-Font 
% \setbeamertemplate{subitem projected}  % Beamer-Color/-Font 
% \setbeamertemplate{subsubitem}  % Beamer-Color/-Font 
% \setbeamertemplate{subsubitem projected}  % Beamer-Color/-Font 
% -------------------------------
%%% 11.2 Hilighting
% \setbeamertemplate{structure}  % Beamer-Color/-Font 
% \setbeamertemplate{local structure}  % Beamer-Color/-Font 
% \setbeamertemplate{tiny structure}  % Beamer-Color/-Font 
% \setbeamertemplate{structure begin} % Beamer-Template
% \setbeamertemplate{structure end} % Beamer-Template
% \setbeamertemplate{alerted text} % Beamer-Color/-Font 
% \setbeamertemplate{alerted text begin} % Beamer-Template
% \setbeamertemplate{alerted text end} % Beamer-Template

%                                                                         BOXES 
%------------------------------------------------------------------------------

%Colores
% \setbeamercolor{upcol}{bg=verdecmyk,fg=marron!20!negro}
% \setbeamercolor{lowcol}{bg=naranja,fg=black}
% \colorlet{shadow.fg}{verdecmyk}
% \colorlet{shadow.bg}{black}

% Conformo las sombras del bemaer box rounded (No tocar)
%Declaro unas bolitas para redondear el roundedbox
\pgfdeclareradialshading[shadow.fg,shadow.bg]{bmb@shadowball}{\pgfpointorigin}{%
  color(0bp)=(shadow.fg!50!shadow.bg); color(4bp)=(shadow.bg)}
\pgfdeclareradialshading[shadow.fg,shadow.bg]{bmb@shadowballlarge}{\pgfpointorigin}{%
  color(0bp)=(shadow.fg!50!shadow.bg); color(4bp)=(shadow.fg!50!shadow.bg); color(8bp)=(shadow.bg)}
%El shadin propiamentedicho
\pgfdeclareverticalshading[shadow.fg,shadow.bg]{bmb@shadow}{200cm}{%
  color(0bp)=(shadow.bg); color(4bp)=(shadow.fg!50!shadow.bg); color(8bp)=(shadow.fg!50!shadow.bg)}
% Transition line
\pgfdeclareverticalshading[lowcol.bg,upcol.bg]{bmb@transition}{200cm}{%
  color(0pt)=(lowcol.bg); color(2pt)=(lowcol.bg); color(4pt)=(upcol.bg)}

 


%\setbeamertemplate{blocks}[shadow=true,upper=red,lower=green]
%\setbeamertemplate{title page}[default][colsep=-4bp,rounded=true,shadow=true]
% \setbeamertemplate{blocks} % Parent Beamer-Template
 %\setbeamertemplate{blocks}[default] 
%\setbeamertemplate{blocks}[rounded][shadow=true,upper=upcol,lower=lowcol]
%\setbeamertemplate{blocks}[rounded][shadow=true,upper=upcol,lower=lowcol]
% \setbeamertemplate{blocks}[rounded][shadow=false]
% -------------------------------
% \setbeamertemplate{block begin} % Beamer-Template
% \setbeamertemplate{block end} % Beamer-Template
% \setbeamertemplate{block title} % Beamer-Color/-Font 
% \setbeamertemplate{block body} % Beamer-Color/-Font 
% \setbeamertemplate{block alerted begin} % Beamer-Template
% \setbeamertemplate{block alerted end} % Beamer-Template
% \setbeamertemplate{block title alerted} % Beamer-Color/-Font 
% \setbeamertemplate{block body alerted} % Beamer-Color/-Font 
% \setbeamertemplate{block example begin} % Beamer-Template
% \setbeamertemplate{block example end} % Beamer-Template
% \setbeamertemplate{block title example} % Beamer-Color/-Font 
% \setbeamertemplate{block body example} % Beamer-Color/-Font 
% -------------------------------
%%% 11.4 Theorem Environments
% \setbeamertemplate{qed symbol} % Beamer-Template/-Color/-Font 
% -------------------------------
% \setbeamertemplate{theorems} % Parent Beamer-Template 
% \setbeamertemplate{theorems}[default] 
% \setbeamertemplate{theorems}[normal font] 
% \setbeamertemplate{theorems}[numbered] 
% \setbeamertemplate{theorems}[ams style]
% -------------------------------
% \setbeamertemplate{theorem begin} % Beamer-Template 
% • \inserttheoremblockenv This will normally expand to block, but if a theorem that has theorem
%   style example is typeset, it will expand to exampleblock. Thus you can use this insert to decide
%   which environment should be used when typesetting the theorem.
% • \inserttheoremheadfont This will expand to a font changing command that switches to the font
%   to be used in the head of the theorem. By not inserting it, you can ignore the head font.
% • \inserttheoremname This will expand to the name of the environment to be typeset (like “Theo-
%   rem” or “Corollary”).
% • \inserttheoremnumber This will expand to the number of the current theorem preceeded by a
%   space or to nothing, if the current theorem does not have a number.
% • \inserttheoremaddition This will expand to the optional argument given to the environment or
%   will be empty, if there was no optional argument.
% • \inserttheorempunctuation This will expand to the punctuation character for the current envi-
%   ronment. This is usually a period.
% -------------------------------
% \setbeamertemplate{theorem end} % Beamer-Template 
% -------------------------------
%%% 11.6 Figures and Tables
% \setbeamertemplate{caption} % Beamer-Template/-Color/-Font 
% \setbeamertemplate{caption}[default] typesets the caption name (a word like “Figure” or “Abbildung” or “Table”)
% \setbeamertemplate{caption}[numbered] adds the figure or table number to the caption. 
% \setbeamertemplate{caption}[caption name own line] 
% -------------------------------
% \setbeamertemplate{caption name} % Beamer-Color/-Font 
% -------------------------------
%%% 11.10    Abstract
% \setbeamertemplate{abstract} 		 % Beamer-Color/-Font 
% \setbeamertemplate{abstract title} % Beamer-Template/-Color/-Font 
% \setbeamertemplate{abstract begin} % Beamer-Template 
% \setbeamertemplate{abstract end}   % Beamer-Template 
% -------------------------------
%%% 11.11 Verse, Quotations, Quotes
% \setbeamertemplate{verse} 		 % Beamer-Color/-Font 
% \setbeamertemplate{verse begin} % Beamer-Template 
% \setbeamertemplate{verse end}   % Beamer-Template 
% \setbeamertemplate{quotation}   % Beamer-Color/-Font 
% \setbeamertemplate{quotation begin} % Beamer-Template 
% \setbeamertemplate{quotation end}   % Beamer-Template 
% \setbeamertemplate{quote}       % Beamer-Color/-Font 
% \setbeamertemplate{quote begin} % Beamer-Template 
% \setbeamertemplate{quote end}   % Beamer-Template 
% -------------------------------
%%% 11.12 Footnotes
% \setbeamertemplate{footnote} % Beamer-Template/-Color/-Font 
% \setbeamertemplate{mark}     % Beamer-Color/-Font footnote 
% -------------------------------
%%% 18.1 Specifying Note Contents
% \setbeamertemplate{note page} % Beamer-Template/-Color/-Font 
% \setbeamertemplate{note page}[default] 
% \setbeamertemplate{note page}[compress] 
% \setbeamertemplate{note page}[plain] 
% -------------------------------
%%% Specifying Which Notes and Frames Are Shown
% \setbeameroption{hide notes}
% \setbeameroption{show notes}
% \setbeameroption{show notes on second screen= location }
% \setbeameroption{show only notes}

% Colores
% =======

% \mode<presentation>{\colorlet{resalta1}{green!70!blue}} 
% \mode<article>{\colorlet{resalta1}{green!65!purple}}
% \mode<presentation>{\colorlet{resalta2}{orange}} 
% \mode<article>{\colorlet{resalta2}{orange!80!bg}}
%
% \setbeamercovered{
%   still covered={\opaqueness<1->{50}},
%   again covered={\opaqueness<1->{50}}
% }
%  

\usefonttheme[onlymath]{serif}
