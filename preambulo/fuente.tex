\usepackage{ulem}
\usepackage{textcomp}
\usepackage{bbding}

% PROBLEMA DE VISUALIZACION DE LOS ACENTOS
%
% El problema aparece cuando se utiliza un tipo de letra que no utiliza la
% codificación T1, y desaparece cuando se usa.
% 
% El tipo por omisión de LaTeX es el Computer Modern que no contiene vocales
% acentuadas. Cuando el texto usa por ejemplo á, LaTeX crea esta letra
% juntando dos caracteres, una a y un acento agudo. El aspecto en pantalla y
% en papel es el correcto, pero al seleccionar ese texto en el Acrobat Reader,
% lo que se obtiene es la secuencia 'a.
% 
% Otros tipos de letra sí contienen una letra á, y si se usa este tipo y se
% indica a LaTeX que use la codificación T1, la selección de texto en Reader
% ya funcionará bien. Así que el problema se soluciona siguiendo estos pasos:
% 
% Indicarle a LaTeX que use la codificación T1.  
% Disponer de una fuente que realmente use esa codificación.
% 
% Si ponemos \usepackage[T1]{fontenc} sin más en un documento, entonces LaTeX
% cambia el tipo por omisión, y en vez de Computer Modern, usará European
% Computer Modern (EC). Si este tipo está instalado, el PDF se generará sin
% problemas y el asunto de seleccionar texto quedará arreglado. Pero aparecerá
% otro problema, y es que normalmente los tipos EC que vienen instaladas en
% las distribuciones de LaTeX vienen sólo en formato PK, que es un formato que
% después se ve muy mal en el Reader (como borroso y descolocado).
% 
% Para esto la solución típica solía ser usar el paquete ae, el cual instala
% un tipo virtual, que aparentemente usa la codificación T1 (para que LaTeX
% quede contento), pero que en realidad usa las fuentes CM en vez de las EC
% (para poder usar los tipos postscript en vez de los PK y así tener un PDF
% que se vea bien con el Reader). Sin embargo esta solución hace que el
% problema de la selección de texto reaparezca, ya que, aunque desde el punto
% de vista de LaTeX los tipos ae usan codificación T1, la realidad es que en
% el PDF final no lo usa, y la á sigue siendo una combinación de a y ' (como
% no podía ser de otra forma, ya que la fuente CM no tiene el carácter á).
% 
% Así que en este caso el truco de usar el paquete ae no sirve, y no queda más
% remedio que recurrir a un tipo que realmente use la codificación T1, como
% los tipos Postscript estándar (Times-Roman, Palatino, etc.)
% 
% Existe una versión postscript de los tipos EC, cuyo desarrollo tiene lugar
% junto al del lenguaje de descripción musical lilypond y están disponibles en
% http://www.lilypond.org/download/fonts/. Asimismo los tipos cm-super,
% disponibles en CTAN:fonts/ps-type1/cm-super/ también aceptan la codificación
% T1.
% 
% Resumiendo, la solución es bien buscar y usar los tipos EC que esten bien
% renderizados o alguno que utilice la codificación T1, o si no compensa el
% esfuerzo utilizar directamente los tipos Postscript estándar. Para ello basta
% añadir al preámbulo del documento

% \usepackage{times}          % Usar tipo Times-Roman, o
% \usepackage{palatino}       % Usar pallatino

\usepackage[T1]{fontenc}    % Usar la codificación T1

% Con esto ya funcionará lo de cortar y pegar desde el Reader, aunque
% utilizando otro tipo de letra con otra métrica que puede cambiar la
% maquetación. En vez de times también pueden utilizarse palatino o bookman,
% dependiendo de los gustos.



% COMO LA CLASIFICO???
% fontfamily                   
% Times New Roman (ptm)        
%\usepackage{textcomp}	 % additional symbols (Text Companion font extension)
%\usepackage{mathptmx}              %% --- Times mit Matheschriften
%\usepackage{bera}
%\usepackage{tpslifonts}            %% --- (Font for Slides)
%\usepackage{mathpazo}              %% --- Palantino
%\usepackage{avant}      	     %% --- Avantgard
%\usepackage[scaled=0.9]{luximono}   %% --- Luxi Mono



%% DESCONOSCO QUE HACE ESTO
%\usepackage{txfonts}

%CAMBIO TAMAÑO
% Font size: The default font size is 10 points. Similarly, you can change the
% default font size for the document to 11 points or 12 points by changing the
% top line to read
% 
%    \documentclass[12pt]{amsart}
% 
% If you want the default text to be a size other than the standard sizes of 10,
% 11, or 12 points (72pt = 1 inch high capital letters) 
%\usepackage{scalefnt}
%\AtBeginDocument{\scalefont{2}}

% Size font for sections
%\usepackage{sectsty}
%\sectionfont{\color{red}\scalefont{2}}
%\subsectionfont{\color{green!80!black}\scalefont{2}}
%\subsubsectionfont{\color{blue!50!white}\scalefont{2}}
  
% Manually adjust each font type
% \renewcommand{\tiny}{\fontsize{12}{14}\selectfont}
% \renewcommand{\scriptsize}{\fontsize{14.4}{18}\selectfont}   
% \renewcommand{\footnotesize}{\fontsize{17.28}{22}\selectfont}
% \renewcommand{\small}{\fontsize{20.74}{25}\selectfont}
% \renewcommand{\normalsize}{\fontsize{24.88}{30}\selectfont}
% \renewcommand{\large}{\fontsize{29.86}{37}\selectfont}
% \renewcommand{\Large}{\fontsize{35.83}{45}\selectfont}
% \renewcommand{\LARGE}{\fontsize{43}{54}\selectfont}
% \renewcommand{\huge}{\fontsize{51.6}{64}\selectfont}
% \renewcommand{\Huge}{\fontsize{61.92}{77}\selectfont}
% \newcommand{\veryHuge}{\fontsize{74.3}{93}\selectfont}
% \newcommand{\VeryHuge}{\fontsize{89.16}{112}\selectfont}
% \newcommand{\VERYHuge}{\fontsize{107}{134}\selectfont}
                                                           

%TIPOS DE FUENTE
% En este catalog trato de poner el paquete, el nombre y los codigos entre parentesis
% de las fuetnes que fui usando, para que luego sea mas facil reusarlas.
% Es extracto de wikipedia y "The LaTeX Font Catalogue", viendo ahi se puede
% ver la muestra. 

% Existen 2 tipografias fundamentales:
% Serif
%   Se refiere a pequeñas semi estructuras a la terminacion de cada letra. La
%   sabiduria convencional dice que estas semi estructuras guian al ojo a
%   travez de la linea (entiéndase aqui renglón) en los cuerpos de texto.
%   
% Sans Serif
%   En frances ``sans'' es ``sin'', es utilizada mas comunmente para
%   encabezados, aunque se usa en los cuerpos de texto en Europa y se volvio la
%   letra ``de facto'' de los cuerpos de texto de internet, dado que se
%   provocan ``twist'' de la serif cuando la usan dispositivos digitales chotos.
%
% Despues hay muchas otras como, tipo maquina de escribir, o escritura a mano, o etc..


%Serif Fonts
%
% Antiqua 
% Antykwa Poltawskiego 
% Antykwa Torunska 
% Antykwa Torunska Condensed 
% Antykwa Torunska Light 
% Antykwa Torunska Light Condensed 
% Bera Serif 
% Boisik 
% Bookman                    (pbk)           
% \usepackage{bookman}               %% Bookman (lädt Avant Garde !!)
% Charter BT 
% \usepackage{charter}               %% --- Charter
% Computer Concrete 
% Computer Modern            % FUENTE DEFAULT DE LATEX!!!! (knuth's)
% Computer Modern Dunhill    (cmdh)          
% Computer Modern Roman      (cmr) 
% Computer Modern Fibonacci  (cmfib)               
% Covington 
% Day Roman 
% Day Roman S 
% Efont Serif 
% GFS Artemisia 
% GFS Artemisia with Euler math 
% GFS Bodoni 
% GFS Didot 
% Garamond 
% KP Serif 
% Kerkis 
% Latin Modern 
% \usepackage{lmodern}               %% --- Latin Modern
% Linux Libertine 
% Literaturnaya 
% New Century Schoolbook     (pnc)            
% \usepackage{newcent}               %% New Century Schoolbook (lädt Avant Garde !!)
% Nimbus Roman 
% Palatino                   (ppl)
% Pandora 
% TeX Gyre Bonum 
% TeX Gyre Pagella 
% TeX Gyre Schola 
% TeX Gyre Termes 
% Times 
% Utopia Regular with Fourier 
% Utopia Regular with Math Design 
% Venturis 
% Venturis No2 
% Venturis Old  
%
% \renewcommand{\rmdefault}{pasx}     % Adobe Aldus
% \usepackage[scaled=1.05]{xagaramon} % Adobe Garamond
% \renewcommand{\rmdefault}{pegx}     % Adobe Stempel Garamond
% \renewcommand{\rmdefault}{pml}      % Adobe Melior
% \renewcommand{\rmdefault}{pmnx}     % Adobe Minion
% \renewcommand{\rmdefault}{psbx}     % Adobe Sabon
% \renewcommand{\rmdefault}{lch}      % Linotype ITC Charter
% \renewcommand{\rmdefault}{lmd}      % Linotype Meridien
 

%Sans Serif Fonts
% Para usar el default de latex de esta tipografia. Como se explico
% arriba puede ser util para documentos planeados para dispositivos.
% \renewcommand{\familydefault}{\sfdefault}. 
%
% Arev 
% Arial 
% Avantgarde                             (pag)
% Bera Sans 
% Computer Modern Bright 
% \usepackage{cmbright}              %% --- CM-Bright (eigntlich eine Familie)
% Computer Modern Sans Serif             (cmss)
% Computer Modern Sans Serif Quotation 
% Courier                                (courier)
% \usepackage{courier}   
% \renewcommand{\familydefault}{\sfdefault}  % Sans Serif Family
% Cyklop 
% Epigrafica 
% GFS Neohellenic 
%
% \usepackage[scaled=0.90]{frutiger}  % Adobe Frutiger
% \usepackage[scaled=0.94]{futura}    % Adobe Futura (=Linotype FuturaLT)
% \usepackage{gillsans}               % Adobe Gill Sans
% \renewcommand{\sfdefault}{pmy}      % Adobe Myriad  
% \usepackage[scaled]{asyntax}        % Syntax
% \usepackage[medium]{optima}         % Adobe Optima (Semi Sans Serif font
% \renewcommand{\sfdefault}{lo9}      % Linotype ITC Officina Sans
%  
% \usepackage[scaled=.90]{helvet}  % Helvetica (phv) -- Sims Arial
% \fontfamily{phv}\selectfont % Letra Arial tamaño 11
% Iwona 
% Iwona Condensed 
% Iwona Light 
% Iwona Light Condensed 
% KP Sans-Serif 
% Kurier 
% Kurier Condensed 
% Kurier Light 
% Kurier Light Condensed 
% LX Fonts 
% Latin Modern Sans 
% Latin Modern Sans Extended 
% Libris ADF 
% Malvern 
% Nimbus Sans 
% Optima 
% Pandora Sans 
% Tapir 
% TeX Gyre Adventor 
% TeX Gyre Heros 
% URW Grotesk 
% Universal 
% Venturis Sans  


%Typewriter Fonts % EL ESTILO DE LAS MAQUINAS DE ESCRIBIR
%
%      Ascii 
%      Bera Mono 
%      CM Pica 
%      Computer Modern Teletype 
%      Computer Modern Teletype L 
%      Computer Modern Typewriter            (cmtt)               
%      Computer Modern Typewriter Proportional 
%      Courier          (pcr)
%      Inconsolata 
%      KP Monospaced 
%      Latin Modern Typewriter 
%      Latin Modern Typewriter Proportional 
%      Letter Gothic 
%      LuxiMono 
%      OCR-A Optical Character Recognition Font A 
%      OCR-B Optical Character Recognition Font B 
%      Pandora Typewriter 
%      TXTT 
%      TeX Gyre Cursor 
%
% 
%Calligraphical and Handwritten Fonts
%
%      Augie 
%      Auriocus Kalligraphicus 
%      BrushScriptX-Italic 
%      Calligra 
%      French Cursive 
%      JD 
%      Jana Skrivana 
%      Lateinische Ausgangsschrift 
%      Lukas Svatba 
%      Osterreichische Schulschrift 
%      PV Script 
%      Schwell 
%      Skeetch 
%      Sutterlin 
%      TW Cal 14 
%      Tall Paul 
%      TeX Gyre Chorus 
%      Teen Spirit 
%      Vereinfachten Ausgangsschrift 
%      Vicentino 
%      Zapf Chancery 
%
% 
%Uncial Fonts
%
%      Artificial Uncial 
%      Carolingan Miniscules 
%      Eiad 
%      Half Uncial 
%      Humanist 
%      Insular Majuscule 
%      Insular Minuscule 
%      Roman Rustic 
%      Rotunda 
%      Square Capitals 
%      Uncial 
%
% 
%Blackletter Fonts
%
%      Early Gothic 
%      Fraktur 
%      Fraktur 
%      Gothic Textura Prescius 
%      Gothic Textura Quadrata 
%      Gotik 
%      Hershey Old English Font 
%      Schwabacher 
%
% 
%Other Fonts
%
%      A Picture Alphabet 
%      Computer Modern Funny Roman 
%      Computer Modern Gray 
%      Decadence 
%      Flyspec 
%      Intimacy 
%      Movieola 
%      Necker 
%      Pookie 
%      Punk 
%      Segment Font 
%      Simfon 
%      Spankys Bungalow 
%      Webster 
%
% 
%Uppercase fonts
%
%      Capital Baseball 
%      Durer 
%      Durer Informal 
%      Durer Sans Serif 
%      Durer Typewriter 
%      FoekFont 
%      Logo 
%      Pacioli 
%      Trajan 
%
% 
%Decorative Initials
%
%      Acorn Initials 
%      Ann Stone 
%      Art Nouveau Caps 
%      Art Nouveau Initialen 
%      Baroque Initials 
%      Carrick Caps 
%      Eichenlaubinitialen 
%      Eileen Caps Black 
%      Eileen Caps Reguler 
%      Elzevier Caps Regular 
%      Gotische Initialen 
%      Goudy Initialen 
%      Kinigstein Caps 
%      Konanur Kaps 
%      Kramer Regular 
%      Morris Initialen 
%      Nouveau Drop Caps 
%      Romantik 
%      Rothenburg Decorative 
%      Royal Initialen 
%      San Remo 
%      Starburst Regular 
%      Typographer Caps 
%      Zallman Caps 

 

%Font Attributes------------
%Under the new font selection system (NFSS) employed by LaTeX2e, a font has five
%attributes: encoding, family, series, shape, and size. 
%LaTeX typesets using a particular encoding vector and a particular font metric
%scaled to a particular size depending upon the values of these attributes. 

%Specify the font attribute:
%  \fontencoding{T1}
%  \fontfamily{garamond}
%  \fontseries{m}    %Medium weight
%  \fontshape{it}    %Italic
%  \fontsize{12}{15} %12pt type with 15pt leading in the T1 encoding scheme
%  \selectfont       %causes LaTeX to look in its mapping scheme for a metric corresponding to these attributes.


%Algunas variables intrinsecas de latex.
%  variable              value       activated by
%  \encodingdefault      OT1         \normalfont, \textnormal{}
%  \familydefault    \rmdefault      \normalfont, \textnormal{}
%  \rmdefault            cmr         \rmfamily, \textrm{}
%  \ttdefault            cmtt        \ttfamily, \texttt{}
%  \sfdefault            cmss        \sffamily, \textsf{}
%  \seriesdefault        m           \normalfont, \textnormal{}
%  \mddefault            m           \mdseries, \textmd{}
%  \bfdefault            bx          \bfseries, \textbf{}
%  \shapedefault         n           \normalfont, \textnormal{}
%  \updefault            n           \upshape, \textup{}
%  \itdefault            it          \itshape, \textit{}
%  \scdefault            sc          \scshape, \textsc{}
%  \sldefault            sl          \slshape, \textsl{}

%  La columna activated by se puede entender con un ejemplo: the \normalfont
%  command (essentially) executes the command sequence:
%    \fontencoding{\encodingdefault}
%    \fontfamily{\familydefault}
%    \fontseries{\seriesdefault}
%    \fontshape{\shapedefault}
%    \selectfont

%The values of these attribute variables may be changed using \renewcommand, for
%example \renewcommand{\familydefault}{\sfdefault} causes the entire document to
%be set in the default sans serif font. Changes to these defaults should be made
%in the document preamble or in a package.


%Change a font---------------------------------

%LaTeX expects three font families as defaults.
%        Font Family                                            Code    Command
%Roman (serif, with tails on the uprights) as the default        rm      \textrm{text}
%Sans-serif, with no tails on the uprights                       sf      \textsf{text}
%Monospace (fixed-width or typewriter)                           tt      \texttt{text}


%Uno puede cambiar momentaneamente una fuente asi:
%{\fontfamily{phv}\selectfont Helvetica looks like this}
%and {\fontencoding{OT1}\fontfamily{ppl} Palatino looks like this}.
%}


%ESCORIA QUE NO SE PARA QUE

  %\mode<article>{ 
  %\usepackage{sectsty} ! no se que es esto
  %\sectionfont{\normalsize}
  %}


%ALGO QUE NO CREO USAR POR AHORA: DISEÑO DE FUENTES...creo

  %Encoding Vectors
  %An encoding vector is a set of instructions to LaTeX detailing how particular
  %symbols are to be constructed. For example, the T1 encoding vector file
  %T1enc.def contains the following commands, among many others:

  %  \DeclareFontEncoding{T1}{}{}
  %  \DeclareTextAccent{\'}{T1}{1}
  %  \DeclareTextSymbol{\ae}{T1}{230}
  %  \DeclareTextComposite{\"}{T1}{a}{228}
    

  %which declare that a T1 encoding vector exists, that, when this encoding vector
  %is in effect, the \'{x} command should superimpose character 1 (accent acute)
  %over the character x, that \ae should produce character 230 (the æ ligature),
  %and that \"a should produce character 228 (the umlaut ä). Notice that an
  %encoding vector implicitly associates a number with each character. The
  %correspondence between characters and numbers used by TeX need not correspond
  %with the correspondence defined by your platform or in your font; don't worry
  %about that, yet.

  %When a user requests a new encoding ENC, LaTeX looks for the file ENCenc.def,
  %where it expects the encoding vector to be defined. You can read more about
  %encoding vectors in the LaTeX companion, but initially you will likely want to
  %use the standard T1 vector; the file T1enc.def should be a part of any LaTeX2e
  %distribution, but for your convience we have also provided a link.

  %Obviously, for a particular character to be accessable in TeX, it must appear
  %in the encoding vector. The authors of the T1 standard chose to include a wide
  %variety of symbols and characters which are used to typeset many languages;
  %they may not, however, have included every character defined in your particular
  %font: yen and copyright symbols, for example, are commonly included in
  %postscript fonts, but do not appear in the T1 encoding. Conversely, the T1
  %encoding may include characters not present in your particular font: most
  %postscript fonts, for example, do not include the ff ligature, which occupies
  %position 27 in the T1 encoding. The absence from your font of a character
  %defined in T1 is not usually a problem; as long as you do not use that
  %character, no difficulties will arise (if you do use a character in your
  %document which is undefined in your font, it will simply appear as a blank
  %space when viewed or printed). Conversely, the absence from the T1 encoding of
  %characters which do exist in your font is a problem, inasmuch as you want to
  %employ these characters. In that case you must define an encoding which
  %includes all the characters you want. The encoding defined in LY1enc.def, used
  %by Y&Y TeX systems, allows access to all the characters in a standard Adobe
  %roman font.

  %Mapping schemes
  %Once a the relevent encoding vector is defined, you will want to set up a
  %scheme which maps font attribute sets to font metrics. For example, the file
  %T1garamond.fd contains the commands

  %  \DeclareFontFamily{T1}{garamond}{}
  %  \DeclareFontShape{T1}{garamond}{m}{n}{ <-> garrm }{}
  %  \DeclareFontShape{T1}{garamond}{m}{it}{ <-> garit }{}

  %which declare the existence of the garamond family with T1 encoding, and say
  %that, when a T1 garamond medium normal font is requested, LaTeX should use the
  %garrm.tfm metric scaled to the appropriate size, while the italic-shape font
  %request used in our earlier example should call up the garit.tfm metric file.
  %When a user requests an unknown font family FAMILY with encoding ENC, LaTeX
  %look for the file ENCFAMILY.fd, where it expects to find a mapping scheme
  %defined. You can read in more detail about font family decleration commands in
  %the LaTeX Companion.

  %To prepare a font family in which the typical LaTeX document can be typeset,
  %you will need to map at least the medium-normal (m-n), medium-italic (m-it),
  %and boldextended-normal (bx-n) series-shape combinations to extant font
  %metrics, in order to set the body text, emphasized text, and section headings,
  %respectively. Some other fonts that documents might request include:
  %medium-smallcaps (m-sc) and medium-slanted (m-sl).

  %Family style files Most of the time, when you prepare a font family for use
  %with LaTeX, you will want to prepare a short file containing commands like

  %  \renewcommand{\encodingdefault}{T1}
  %  \renewcommand{\rmdefault}{garamond}

  %Call it garamond.sty and place it in a directory searched by TeX so that users
  %can call call it up simply by issuing the command \usepackage{garamond} in the
  %document prologue, and have the entire document set in garamond without further
  %ado.

  %When setting up such a style file, you might also want to include declarations
  %of accompanying \sffamily and \ttfamily fonts which are visually compatible
  %with your \rmdefault font.








% OTRAS COSAS

% \usepackage{relsize} % Tamaño de fuente relativo mediante comando \relsize{#1}

