
% CODIFICACION

%\makeatletter
%\newcommand\thefontsize[1]{{#1 The current font size is: \f@size pt\par}}
%\makeatother

%\usepackage[latin1]{inputenc} % Para UNIX iso-8889-1 (también conocida como latin1).
\usepackage[utf8]{inputenc}  % utf-8, como en muchas distribuciones recientes de Linux
%\usepackage{ucs}  soporte para utf-8 distro antigua de Linux
%\usepackage[utf-8]{inputenc}
%\usepackage[cp1252]{inputenc} % Para MS-Windog cp1252 (que es casi el latin1)
%\usepackage[cp850]{inputenc}  % Para DOS
%\usepackage[applemac]{inputenc}  % Para Macintosh (aunque hay editores de Mac que guardan como latin1 o utf8).
 
% Errores de unicode!
%Una forma de buscar caracteres extraños en vim es /[^\x00-\x7F] Luego
%``ga'' sobre el caracter deseado brinda mas información  y luego usar la
%representacion hexadecimal 00NN. Tambien sirve para buscar de la forma
%/<ctrl-v>xNN
%Igual hay caracteres muy dificiles de encontrar, entonces se recomienda:
%Unhappily utf8.def does not show the numerical representation for the missing
%Unicode character. The missing character <char> is shown directly in macro
%\u8:<char>. The following example adds the numerical information in the error
%message:
\usepackage{stringenc}
\usepackage{pdfescape}
\makeatletter
\renewcommand*{\UTFviii@defined}[1]{%
  \ifx#1\relax
    \begingroup
      % Remove prefix "\u8:"
      \def\x##1:{}%
      % Extract Unicode char from command name
      % (utf8.def does not support surrogates)
      \edef\x{\expandafter\x\string#1}%
      \StringEncodingConvert\x\x{utf8}{utf16be}% convert to UTF-16BE
      % Hexadecimal representation
      \EdefEscapeHex\x\x
      % Enhanced error message
      \PackageError{inputenc}{Unicode\space char\space \string#1\space
                              (U+\x)\MessageBreak
                              not\space set\space up\space
                              for\space use\space with\space LaTeX}\@eha
    \endgroup
  \else\expandafter
    #1%
  \fi
}
\makeatother

\DeclareUnicodeCharacter{2215}{/}
%\DeclareUnicodeCharacter{fc}{\"u}
%\DeclareUnicodeCharacter{0252}{\"u}
%\DeclareUnicodeCharacter{0169}{\~u}
\DeclareUnicodeCharacter{00A0}{ }
\DeclareUnicodeCharacter{2210}{-}
\DeclareUnicodeCharacter{2212}{--}
\DeclareUnicodeCharacter{2013}{--}
\DeclareUnicodeCharacter{2243}{$\simeq$}
%\DeclareUnicodeCharacter{00A0}{~} ! NO, necesito controlar el brake line
\DeclareUnicodeCharacter{0338}{ü}
\DeclareUnicodeCharacter{0214}{ö}
%\DeclareUnicodeCharacter{0144}{ń} 
\DeclareUnicodeCharacter{2082}{$_2$}
\DeclareUnicodeCharacter{00A9}{$\copyright$}
\DeclareUnicodeCharacter{00D7}{x}
\DeclareUnicodeCharacter{2018}{'}
\DeclareUnicodeCharacter{2018}{´}
\DeclareUnicodeCharacter{2032}{´}
\DeclareUnicodeCharacter{00B1}{$\pm$}
\DeclareUnicodeCharacter{00B0}{$^o$}
\DeclareUnicodeCharacter{0142}{\l}
\DeclareUnicodeCharacter{0131}{í} %acento en dos tiempos
\DeclareUnicodeCharacter{0301}{}  %acento en dos tiempos
\DeclareUnicodeCharacter{0300}{}  %acento en dos tiempos
\DeclareUnicodeCharacter{221A}{$\sqrt$}
\DeclareUnicodeCharacter{221E}{$\infty$}
\DeclareUnicodeCharacter{2192}{$\rightarrow$}
\DeclareUnicodeCharacter{2215}{/}
\DeclareUnicodeCharacter{201C}{``}
\DeclareUnicodeCharacter{201D}{''}
                        
