% When including graphics, if not extension is given, the order of preference
% when files with the same name and different extensions is
% .png .pdf .jpg .mps .jpeg .jbig2 .jb2 .PNG .PDF .JPG .JPEG .JBIG2 .JB2
% Below we can change the preference for pdf first
% \DeclareGraphicsExtensions{
%   .pdf,.png,.jpg,.mps,.jpeg,.jbig2,.jb2,
%   .PDF,.PNG,.JPG,.JPEG,.JBIG2,.JB2}
% See also http://tex.stackexchange.com/a/45502/28411 



% CAPTIONS IN TABLES AND FIGURES

% Siguiendo http://tex.stackexchange.com/a/2652/28411 agrego esto para el
% centrado automatico de figuras
%\makeatletter
%\g@addto@macro\@floatboxreset\centering
%\makeatother

% Floatrow environment, automatic centering, caption position and width
% \usepackage{floatrow}
% \floatsetup[table]{style=plaintop} % put the caption of tables above
% Some usefull commads are \fcapside, \fcapside[\FBwidth]
% Use 
% \begin{table} 
%   \begin{floatrow} 
%     \ttabbox
%     {\caption{\ldots}\label{\ldots}}
%     {\ldots}
%     \ttabbox
%     {\caption{\ldots}\label{\ldots}}
%     {\ldots}
%   \end{floatrow}
% \end{table}
% This package can give unexpected behavior in the float in not so frequent
% uses. In that case the command \RawFloats will avoid the control of that
% float by the package




\RequirePackage[
   font=small,
   %format=plain,
   singlelinecheck=off,
   labelfont=bf,
   %textfont=,
   %justification=centering,
   %justification=justified,
   justification=centerlast,
   up]{caption}
       

\RequirePackage{epsfig} %manejo de figuras eps
\RequirePackage{sidecap} %caption in teh sides

%\RequirePackage{floatpag} %Estilo de pagina de figuras
%%\thisfloatpagestyle{empty} inside a figure/table environment clean the floatpage's style
%\rotfloatpagestyle{empty} % to afect all sideways figure or tables (rotating package)

\RequirePackage{lscape} % Pagina para leer en el lado largo, ejemplo:
%\begin{landscape}
%asd
%\end{landscape}

\RequirePackage{rotating} % Flotante para leer en el lado largo, ejemplo:
%\begin{sidewaysfigure}
%asd
%\end{sidewaysfigure}


  

%\RequirePackage{capt-of} %caption fuera de figure
%use: \captionof{figure}{me cacho en deiz} 

% Preferiblemente en un ambiente (i.e. center)

\RequirePackage{float} %anclar los flotantes
%use: \begin{figure}[H]

\RequirePackage{wrapfig} %figuras ancladas al lado del parrafo siguiente
%use: \begin{wrapfigure}{r}{0.45\textwidth}
\columnsep0.7cm %Separacion de columnas entre warpfig y texto


%\RequirePackage[rflt]{floatflt} %figuras flotantes al lado del texto
%use: \begin{floatingfigure}[r]{0.45\textwidth}

%\RequirePackage{subfig} %fotos con caption individual en una figura
%use: \begin{figure}
%     \subfloat[]{\includegraphics[]{}}

\RequirePackage[%
   %final,
   %draft % no incluye las imagenes (rapido)
]{graphicx}

\RequirePackage{graphics}


%                                                                 TABLAS
%-----------------------------------------------------------------------

\usepackage{rotating} %para escribir en vertical
\usepackage{multirow} %para escribir en vertical en las tablas
% \usepackage{tabularx}   % Erweiterte Tabellen Optionen
% \usepackage{booktabs}
% \usepackage{multicol}


\newcommand{\vc}[3]{\multirow{#1}{*}[-1mm]{\rotatebox{#2}{#3}}}
\newcommand{\mr}[3]{\multirow{#1}{#2}{#3}}

%Tablas
\newcommand{\mc}[3]{\multicolumn{#1}{#2}{#3}}

%\usepackage{booktabs}


%The width and height of a cell in a tabular is controlled by many parameters: 
  \renewcommand{\tabcolsep}{0.5cm}
  \renewcommand{\arraystretch}{1.3}
%\arraystretch is part of the LaTeX format it multiplies the height and depth of the "strut" used to space out table rows by the specified factor.
%\extrarowheight is an extra parameter added by the array package which adds a specified length to the height of the strut used for padding table rows.
%It's the difference between adding and multiplying.
%If you want to keep text away from horizontal lines without disturbing everything else too much \extrarowheight is usually better.
%  \begin{tabular}{|p{2cm}|p{3cm}|}
%  \hline
%  c & c \\
%  \hline
%  c & c \\
%  \hline
%  \end{tabular}
%
%You specify p{2cm} as part of your \begin{tabular}{} arguments, in the hope
%that the first column in the table is 2cm wide. Unfortunately, it appears wider
%than you think. The width of a cell in the table, without regard to the line
%width of separators, is actually computed by
%
%\tabcolsep + p{length} + \tabcolsep.
%
%The length you specify, gives place to contain characters in the cell. Between
%the left separator and the left side of the bounding box of the first
%character, there is some room which is controlled by \tabcolsep, such that the
%cell will not look too crowded. It is the same on the right side of the cell.
%In other words, \tabcolsep governs such that the contents in the cell will not
%be positioned right next to the boarders, which looks rather ugly.
%
%By default, \tabcolsep is set to 6pt, which equals to 2.12mm in digital
%printing. In the above codes, we re-set it to 1cm. So the total width of the
%first column in the table is 4cm, while the second column is 5cm wide.
%
%The mechanism of the height of a cell is a little bit different. In default
%setting, the distance between the upper boarder and lower boarder of a cell is
%\baselineskip, which is the line spacing in paragraphs. If you look at two
%adjacent lines of texts in the paragraph, \baselineskip is the distance between
%the two base lines of the texts. \baselineskip is specified at the font
%selection stage. The primitive command \fontsize{size}{skip} sets this value.
%Usually a 10pt font size is associated with 12pt line skip.
%
%The command \arraystretch scales the height of the cell by a factor. As in the
%above codes, the spacing of a row in the table is 2 times the default.
%
%Note that the default height of a row in a tabular cannot be changed by
%manually setting \baselineskip. As to my current knowledge, the height of a row
%can only be changed by specifying a different \arraystretch factor. (Similarly,
%if you want to change the line spacing in the texts, such as to double spacing,
%do not change \baselineskip. Use \baselinestretch instead.)


