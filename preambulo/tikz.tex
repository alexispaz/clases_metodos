% Estilos
% =======
% Estos estilos son mas portables si se pegan en cada tikzpicture

% Control `blur shadow`
% every shadow/.style={shadow xshift=-2ex, shadow yshift=-2ex}, 

% El cuadro redondeado
% cuadro/.style={rectangle, rounded corners=1mm,fill=white,inner sep=3,text opacity=1},

% El recorte con tijera random 
% recorte/.style={draw=azul,fill=white,thick,draw, minimum height=2cm, minimum width=3cm,
%        decorate, decoration={random steps,segment length=3,amplitude=2}},

% Flecha filosa
% flecha/.style={->,>=stealth,verde,line width=1mm},

% Teorema-like a mano
% theo/.style={rectangle, rounded corners=1ex,draw=verde,fill=verde,fill opacity=0.2,text opacity=1, anchor=north west,inner sep=6,align=center},
% theohead/.style={theo,inner sep=4,fill opacity=1,anchor=west,font={\bfseries}},

% Otros del monton  
% tit/.style={font=\Large\color{bg},anchor=north west,text width=\textwidth},
% nube/.style={cloud, cloud puffs=10.8,cloud puff arc=110, aspect=2, fill=white
%  ,double,double distance=1mm, inner sep=0mm,align=center,
%  font=\small\color{bg}}
% itema/.style={anchor=north west},
% flecha/.style={ >=latex, ->, shorten <={-1.5} },
% segmento/.style={ ultra thick,
% {Circle[width=4,length=4]}-{Circle[width=4,length=4]}, shorten <={-2},
% shorten >={-2} }, 
% semirecta/.style={ {Circle[width=3,length=3]}-latex, shorten <={-1.5} },
% def/.style={ anchor=north west, align=justify, text=fg, rectangle, rounded
% corners=1ex, thick, draw=verde, % font={\baselineskip=3.6mm}, },
% txt/.style={ anchor=north west, align=justify, text=fg, rectangle, }  

% MANDE A ../TIKZ.BKP ALGUNAS COSAS QUE ME PARECIAN AL VICIO

%
%Se puede usar todas las keys en un \tikzset{}, empieze con tikz o con pgf
%pudiendola escribr completa (i.e./tikz/..) o no.
%si uno escribe mio./style uno esta definiendo un estilo
\RequirePackage{tikz,pgf}

\usepackage{xifthen}

% Fixing the seed
\pgfmathsetseed{\number\pdfrandomseed}

% Avoid some dimension too large (but take longer)
\usepackage{fp}
\usetikzlibrary{fixedpointarithmetic}
 

%\usepackage{setspace} % spacing environment

% To avoid pading around figures in nodes with figures
\tikzset{graphics/.style={inner sep=0,outer sep=0}}

% To scale even linewidth and fonts
%\tikzset{scaleall/.style={scale=#1, every node/.append style={scale=#1}}}
\tikzset{scaleall/.style={scale=#1, transform shape}}

%% Prefix the name of the nodes in a scope
%% http://tex.stackexchange.com/a/128079/28411
%\makeatletter
%\tikzset{%
%  prefix node name/.code={%
%    \tikzset{name/.code={\edef\tikz@fig@name{#1 ##1}}}
%  }%
%}
%\makeatother
%The problem is that you can have to put the prefix when refers to internal
%nodes inside the scope

%\makeatletter
%\tikzset{privatebbox/.style={
%    %execute at begin scope={
%    %    % save the bounding box
%    %    \pgfpointanchor{current bounding box}{south west}
%    %    \pgfgetlastxy\tsx@outerbb@minx\tsx@outerbb@miny
%    %    \pgfpointanchor{current bounding box}{north east}
%    %    \pgfgetlastxy\tsx@outerbb@maxx\tsx@outerbb@maxy
%
%    %    % clear the bounding box
%    %    \pgfresetboundingbox
%    %},
%    %execute at end scope={
%    %    % do something useful with the scope bounding box
%    %    \draw (current bounding box.south west) rectangle (current bounding box.north east);
%
%    %    % reestablish the outer bounding box.
%    %    \path (\tsx@outerbb@minx, \tsx@outerbb@miny) rectangle (\tsx@outerbb@maxx,\tsx@outerbb@maxy);
%    %}
%}}
%\makeatother


% How to proyect into a plane in 3D
%http://tex.stackexchange.com/a/114266/28411



%tikz nots
\newcommand{\tkzn}[1]{\tikz[baseline=(#1.base),remember picture]\node(#1){};}
\newcommand{\tkznt}[2]{\tikz[baseline=(#1.base),remember picture]\node(#1){#2};}
\newcommand{\tkznot}[3]{\tikz[baseline=(#1.base),remember picture]\node(#1)[#2]{#3};}
\newcommand{\tkzpnot}[4]{\tikz[remember picture]\path(#1) node(#2)[#3]{#4};}
 

%use as {\paperwidth}{\paperheight}
\newlength{\bbw}
\newlength{\bbh}
\newcommand{\pcuad}[3][]{
  % First argument is an optional prefix for the coordinate names

  \setlength{\bbw}{#2}
  \setlength{\bbh}{#3}
  \useasboundingbox(0,0) rectangle (\bbw,\bbh);

  \path(.5\bbw,.5\bbh) coordinate(#1cp);
  \path(\bbw,0)        coordinate(#1se);
  \path(0,0)           coordinate(#1sw);
  \path(0,\bbh)        coordinate(#1nw);
  \path(\bbw,\bbh)     coordinate(#1ne);
 
  
  \path(\bbw,.5\bbh) coordinate(#1ep);
  \path(.5\bbw,0)    coordinate(#1sp);
  \path(.5\bbw,\bbh) coordinate(#1np);
  \path(0,.5\bbh)    coordinate(#1wp);
       
  \path(.75\bbw,.5\bbh) coordinate(#1hr);
  \path(.25\bbw,.5\bbh) coordinate(#1hl);
  \path(.5\bbw,.25\bbh) coordinate(#1vl);
  \path(.5\bbw,.75\bbh) coordinate(#1vu);

  \path(.75\bbw,.75\bbh) coordinate(#1c1);
  \path(.25\bbw,.75\bbh) coordinate(#1c2);
  \path(.25\bbw,.25\bbh) coordinate(#1c3);
  \path(.75\bbw,.25\bbh) coordinate(#1c4);


%  +-----------------------------------------------+
%  |nw                    np                     ne|
%  |                                               |
%  |                                               |
%  |          c2          vu          c1           |
%  |                                               |
%  |                                               |
%  |wp        hl          cp          hr         ep|
%  |                                               |
%  |                                               |
%  |          c3          vl          c4           |
%  |                                               |
%  |                                               |
%  |sw                    sp                     se|
%  +-----------------------------------------------+
}
 
        
\newcommand{\showcuad}[1][]{

  \draw[help lines,xstep=.5,ystep=.5,dashed,gray,opacity=0.2] (#1sw) grid (#1ne);
  \draw[help lines,xstep=1,ystep=1,gray,opacity=0.5]      (#1sw) grid (#1ne);

  \foreach \x in {0,2,...,10} {
      \node [anchor=north, gray,opacity=0.2,yshift=30] at (\x,0) {\tiny \bf \x};
      \node [anchor=east,gray,opacity=0.2,xshift=30] at (0,\x) {\tiny \bf  \x};
  }
             
  \fill[gray,opacity=0.2](#1se) circle(.1) node[anchor=south east]{#1se};
  \fill[gray,opacity=0.2](#1sw) circle(.1) node[anchor=south west]{#1sw};
  \fill[gray,opacity=0.2](#1ne) circle(.1) node[anchor=north east]{#1ne};
  \fill[gray,opacity=0.2](#1nw) circle(.1) node[anchor=north west]{#1nw};

  \fill[gray,opacity=0.2](#1hr) circle(.1) node[above]{#1hr};
  \fill[gray,opacity=0.2](#1hl) circle(.1) node[above]{#1hl};
  \fill[gray,opacity=0.2](#1vu) circle(.1) node[above]{#1vu};
  \fill[gray,opacity=0.2](#1vl) circle(.1) node[above]{#1vl};

  \fill[gray,opacity=0.2](#1sp) circle(.1) node[anchor=south]{#1sp};
  \fill[gray,opacity=0.2](#1wp) circle(.1) node[anchor=west] {#1wp};
  \fill[gray,opacity=0.2](#1np) circle(.1) node[anchor=north]{#1np};
  \fill[gray,opacity=0.2](#1ep) circle(.1) node[anchor=east] {#1ep};

  \fill[gray,opacity=0.2](#1cp) circle(.1) node[above]{#1cp};
  \fill[gray,opacity=0.2](#1c1) circle(.1) node[above]{#1c1};
  \fill[gray,opacity=0.2](#1c2) circle(.1) node[above]{#1c2};
  \fill[gray,opacity=0.2](#1c3) circle(.1) node[above]{#1c3}; 
  \fill[gray,opacity=0.2](#1c4) circle(.1) node[above]{#1c4};

  %\fill[red](cp|-c1) circle(.1) node[anchor=north]{cp|-c1};

}
 

%\usepackage{eso-pic} % This give absolut positioning in the page:
%\AddToShipoutPictureBG{}
%  All the picture commands which are parameters of an \AddToShipoutPictureBG
%  command will be added to the internal macro \ESO@HookIBG. This macro is part
%  of a zero-length picture environment with basepoint at the lower left corner of
%  the page (default) or at the upper left corner (package option ”texcoord”). The
%  picture environment will be shipped out at every new page.
%\AddToShipoutPictureBG* works like \AddToShipoutPictureBG but only for
%  the current page. It used the internal macro \ESO@HookIIBG which contents will
%  be removed automatically.

% The problem with this is that only, uncover and hidden of beamer don't work
%\newenvironment{frametikz}[1]{
%\RequirePackage{envrion}
%\NewEnviron{frametikz}{
%  \AddToShipoutPictureFG*\bgroup\put(-3,0)\bgroup \begin{tikzpicture} [overlay,#1] \pcuad
%  \BODY
%  \end{tikzpicture}\egroup\egroup
%}

% This is the chosen one for now
%\begin{frame}\par
%\frametitle<presentation>{Coalescence of Au and Co}
%\hspace*{-2pt}
%\vspace*{-2pt}
%\makebox[\linewidth][c]{
%\begin{tikzpicture}
%\pcuad{\paperwidth}{\paperheight}
%\path<1>(hr)++(1.15,0)  node{\includegraphics[width=0.25\textwidth]{./Imagenes/hddm/diag1.png}};
%\path<2>(hl)++(1.15,0)  node{\includegraphics[width=0.25\textwidth]{./Imagenes/hddm/diag1.png}};
%\end{tikzpicture}}
%\end{center} 
%\end{frame}
 

% It would be nice to have a library to mix tikz with the frames, and do the
% svg trick and also allow a subnode command to change the colors with beamer
% animations.

% Deprecated, see tikzmark
% Include a node for parts of an equation
%\newcommand\mytikzmark[3][]{%
%  \tikz[remember picture,baseline=(#2.base)]{\node(#2)[outer sep=0pt,#1]{#3};}%
%}

% Compatibilidad con babel 
% http://tex.stackexchange.com/a/166775/28411  

  % PGF/TikZ version 2.10
  % \tikzset{
  %   every picture/.append style={
  %     execute at begin picture={\deactivatequoting},
  %     execute at end picture={\activatequoting}
  %   }
  % }

  % PGF/TikZ version 3.0.0
  % \usetikzlibrary{babel}

% LIBRERIAS 
% ---------

% Para usar span
% \usetikzlibrary{positioning-plus}

% Zoom con una lupa. 
% NOTA: Hay que poner `spy using outline` en el tikzpicture
\usetikzlibrary{spy}

% Por alguna razon, spy cambia los colores foreground. Esto lo arregla.
% https://tex.stackexchange.com/a/204296/28411
\makeatletter
  \tikzset{%
    tikz@lib@reset@gs/.style={thin,solid,opaque,line cap=butt,line join=miter}
  }
\makeatother

% Decoración de nodos
\usetikzlibrary{
  decorations,
  decorations.fractals,
  decorations.shapes,
  decorations.text,
  decorations.pathmorphing,
  decorations.pathreplacing,
  decorations.footprints,
  decorations.markings,
}

%\usetikzlibrary{fpu} % Float point unit para operar
\usetikzlibrary{%
arrows,% Mas flechas que las de pgf
calc,% Para operar con coordnadas. Por ejemplo: ($(0,1)+(1,0)$) es (1,1) 
fit,%
patterns,%
plotmarks,%
shapes, % Reconoce la notacion .north .30 en nodos de distintas formas(circulo, elipse, etc)
% shapes.geometric,%
% shapes.misc,%
% shapes.symbols,%
% shapes.arrows,%
% shapes.callouts,%
% shapes.multipart,%
% shapes.gates.logic.US,%
% shapes.gates.logic.IEC,%
tikzmark, %the use of subnodes and tikzmarks, usefull to interconect tikz pictures
er,%?
automata,%?
backgrounds,% Manejo del background canvas del tikz ?
chains,%
topaths,%
trees,% Arboles, childs y esas cosas
petri,%
mindmap,% Arboles y childs con conexiones lindas, como neuronales
matrix,% Para funcionalidades en matrices como matrix of nodes
%calendar,%
folding,%
fadings,%
through,%
%positioning,%
scopes,%
hobby,%John Hobby’s algorithm, smooth curves as a list of cubic Bzier curves with endpoints at subsequent points in the list.
calligraphy,
% Al usar un text, tikz escribe caracter por caracter en un
%hbox. Entonces is se quiere imprimir uno indicandolo con más de uno (e.g.
%ó=\'{o}) es necesario encerrarlos entre braces.
shadows,% Sombras y transparencias
shadows.blur,
external% Externalización de gráficos
}

% LAYERS

% Creo una capa detras de la principal pero arriba del background
% es util para cosas como los diagramas de gantt, para
% que las flechas vayan por detras.
\pgfdeclarelayer{behind}
\pgfsetlayers{background,behind,main}

% To get \begin{scope}[zlevel=main] .....
\tikzset{zlevel/.style={%
        execute at begin scope={\pgfonlayer{#1}},
        execute at end scope={\endpgfonlayer}
}}

% to get \node[on layer=behind]....
%%% see https://tex.stackexchange.com/a/20426
\makeatletter
\pgfkeys{%
  /tikz/on layer/.code={
    \pgfonlayer{#1}\begingroup
    \aftergroup\endpgfonlayer
    \aftergroup\endgroup
  },
  /tikz/node on layer/.code={
    \gdef\node@@on@layer{%
      \setbox\tikz@tempbox=\hbox\bgroup\pgfonlayer{#1}\unhbox\tikz@tempbox\endpgfonlayer\egroup}
    \aftergroup\node@on@layer
  },
  /tikz/end node on layer/.code={
    \endpgfonlayer\endgroup\endgroup
  }
}
\def\node@on@layer{\aftergroup\node@@on@layer}
\makeatother
%%%
             


%                                                               Gantt charts
%------------------------------------------------------------------------------
% To perform Gantt Charts:
% A chart in which a series of horizontal lines shows the amount of work done or
% production completed in certain periods of time in relation to the amount
% planned for those periods.
\usepackage{pgfgantt}

% Esto es una forma de alterar el estilo de ganttbar
% \newganttchartelement{auxbar}{
%   bar/.style={
%   shape={chamfered rectangle},
%   chamfered rectangle corners={north east,south east},
%   inner sep=5pt,
%   draw=cyan!60!black,
%   very thick,
%   top color=white,
%   bottom color=cyan!50,
%   }
% }

\newganttchartelement{code}{
code/.style={
  shape={rectangle},
  inner sep=8pt,
  draw=pink!70!black,
  very thick,
  top color=white,
  bottom color=pink!50,
  },
}  
 
\newganttchartelement{sim}{
sim/.style={
  shape={chamfered rectangle},
  chamfered rectangle corners={north east,south east},
  inner sep=5pt,
  draw=cyan!60!black,
  very thick,
  top color=white,
  bottom color=cyan!50,
  },
  sim label font=\slshape,
  sim left shift=0,
  sim right shift=0,
}
                
\newganttchartelement{exp}{
exp/.style={
  shape={chamfered rectangle},
  chamfered rectangle corners={north east,south east},
  inner sep=5pt,
  draw=verde!60!black,
  very thick,
  top color=verde!50,
  bottom color=white,
  },
}  
               
\newganttchartelement{simexp}{
simexp/.style={
  shape={chamfered rectangle},
  chamfered rectangle corners={north east,south east},
  inner sep=7pt,
  draw=verde!70!cyan!60!black,
  ultra thick,
  top color=verde!50, 
  bottom color=cyan!50,
  },
}  
   

\newganttlinktype{mix}{
\ganttsetstartanchor{south}
\ganttsetendanchor{north}
\path(\xLeft, \yUpper)  ++(0,0.06)coordinate(b);
\path(\xRight, \yLower) ++(0,0.06) coordinate(e);
\path($(b)!.5!(e)$) coordinate(m);
\draw [on layer=behind,<-<,>=stealth,shorten >=-3,ultra thick, red] (b-|m) |- (m|-e) ;
}
 
\newganttlinktype{ston2}{
\ganttsetstartanchor{south}
\ganttsetendanchor{north}
\path(\xLeft, \yUpper)  ++(0,0.06)coordinate(b);
\path(\xRight, \yLower) ++(0,0.06) coordinate(e);
\path($(b)!.5!(e)$) coordinate(m);
\draw [on layer=behind,-<] (b-|m) |- (m|-e) ;
}
       

\newganttlinktype{stow}{
\ganttsetstartanchor{south}
\ganttsetendanchor{west}
  \draw[->,on layer=behind] (\xLeft, \yUpper) |- (\xRight, \yLower) ;
}
 
\newganttlinktype{eton}{
\ganttsetstartanchor{east}
\ganttsetendanchor{north}
  \draw[->,on layer=behind] (\xLeft, \yUpper) -| (\xRight, \yLower) ;
}
                   



         
%                                                               EXTERNALIZACION
%------------------------------------------------------------------------------

% Updated picture message
% \tikzset{external/verbose IO={false}}

%Esta linea es para que los graficos sean actualizados solo cuando sea necesario
%Esto se conoce como externalization graphics, y solo compila una vez los tikz
%picture y luego los importa del pdf adecuado en la carpeta Externos 
%\tikzsetexternalprefix{TikZ/} %%%%% NO, LO PONGO EN EL NAME DIRECTAMENTE

% Esto es para cambiar el nombre de cada imagen externalizada
  %\tikzsetnextfilename{AuPt300msErotEvib}

% This avoid to externalize the tikzpicture environment
\tikzset{/tikz/external/export={false}}

% Resolution of a tikzpicture that uses bitmap files
\def\tikzpngresolution{300 }
% dejar espacio al final!!!!
 
% Definition of the \includetikz.
% Use a separate file to include a long tikzpicture trough this command.
% It is better to have a folder with all this files togehter.
% A counter will increase in each call to this command. For portability, the
% important auxiliary files of the pgfplots and tikz externalization will be
% moved to this folder. Non important auxiliaries files will be deleted.
% The command admit two arguments. The first argmuent select one of the
% following behavior:
% *   this plot will be externalized and will not update any more (fast compilation).
% r   the picture will be updated using the stored input files readed again.
% R   the picture will be updated and all the input files readed again.
% H   hide this plot (fast compilation).
% h   this plot will be externalized and will not update any more (fast compilation).
% png use pdftoppm to include a bitmap of the plot (faster compilation for many data plots)
\def\includetikz[#1]#2{
  {
    % Activa la externalizacion
    \tikzset{external/export={true}} 
    \tikzsetnextfilename{#2}

    % Cuando usamos "raw gnuplot" en pgfplots el script y la tabla para gnuplot tambien
    %\pgfkeys{tikz/prefix={TikZ/}}
    %Contador para los caption
    %\setcounter{gnuplotcounter}{0}
%    \def\externalname{#2}
    

    % FORMA DE CARGA DEL TIKZPICTURE SEGUN ARGUMENTO #1
    \ifthenelse{\equal{#1}{\asterisco}}{
      % (*) NO LA FUERZA NI LA ESCONDE
      \input{#2.tex}
    }{
      \ifthenelse{\equal{#1}{H}}{
      % (H) SOLO LA ESCONDE.. NO PUEDO ESCONDER BOUNDING BOX
        \pgfkeys{/pgf/images/include external/.code={}}
        \input{#2.tex}
      }{
        % (h) LA ESCONDE Y LA FUERZA
        \ifthenelse{\equal{#1}{h}}{
          \tikzset{/tikz/external/mode={only graphics}}
          \input{#2.tex}
        }{
          % (R) REMARK, LA FUERZA Y BORRA LAS TABLAS O PNG SI LAS HUBIERA
          \ifthenelse{\equal{#1}{R}}{
            \tikzset{/tikz/external/force remake}
            \immediate\write18{rm -f #2*.table #2*.png}
            \input{#2.tex}
          }{
            % (r) REMARK, LA FUERZA
            \ifthenelse{\equal{#1}{r}}{
              \tikzset{/tikz/external/force remake}
              \input{#2.tex}
            }{
              % (png) GRAPHIC, UN INCLUDE DEL PDF CONVERTIDO A PNG LA
              % RESOLUCION ES \tikzpngresolution DPI. ESTO ES
              % EQUIVALENTE A * EN QUE NO REMARKEA
              \ifthenelse{\equal{#1}{png}}{
                \immediate\write18{if [ ! -f #20-1.png ]; then pdftoppm -png -r \tikzpngresolution #20.pdf #20; fi}

                \def\oldpdfimageresolution{\pdfimageresolution}
                \pdfimageresolution \tikzpngresolution
                \includegraphics{#20-1.png}
                \pdfimageresolution \oldpdfimageresolution 

                % La idea de esto es porque hay tikz que dan pdf que resulta lentisimo al
                % visualizar (Ejemplo pgfplots 3d graphics).  Porque vuelvo a 72 dpi???
                % porque parece que es la forma de mantener el tamaño en la conversion de
                % manera tal que en el documento final sea lo mismo meter un pdf o un
                % png. Es decir, cuando el latex lee un pdf le pone 72dpi de
                % resolucion... es algo mas complicado pero se vee en este fragmetno
                % googleado:
                
                %See pdftex manual:
                %  \pdfimageresolution (integer)
                %  The integer \pdfimageresolution parameter (unit: dots per inch, dpi) is
                %  a last resort value, used only for bitmap (jpeg, png, jbig2) images,
                %    but not for pdfs. The priorities are as follows: Often one image
                %      dimension (width or height) is stated explicitely in the TEX file. Then
                %      the image is properly scaled so that the aspect ratio is kept. If both
                %      image dimensions are given, the image will be stretched accordingly,
                %    whereby the aspect ratio might get distorted. Only if no image
                %      dimension is given in the TEX file, the image size will be calculated
                %      from its width and height in pixels, using the x and y resolution
                %      values normally contained in the image file. If one of these resolution
                %      values is missing or weird (either < 0 or > 65535), the
                %      \pdfimageresolution value will be used for both x and y > resolution,
                %    when calculating the image size. And if the \pdfimageresolution is
                %      zero, finally a default resolution of 72 dpi would be taken. The
                %      \pdfimageresolution is read when pdfTEX creates an image via
                %      \pdfximage. The given value is clipped to the range 0..65535 [dpi].
                %      Currently this parameter is used particularily for calculating the
                %      dimensions of jpeg images in exif format (unless at least one dimension
                %          is stated explicitely); the resolution values coming with exif files
                %      are currently ignored.
              }{
               }
             }
           }
         }
       }
     }
    %Imprimo el final del log por posibles errores, borro el log
    %\immediate\write18{cat #2#30.log | tail}
    %\immediate\write18{rm -f #2.log}


    \immediate\write18{rm -f #2*.vrb}
    %\immediate\write18{rm -f #2*.gnuplot}
    \immediate\write18{rm -f #2*.toc}
    \immediate\write18{rm -f #2*.nav}
    \immediate\write18{rm -f #2*.out}
    \immediate\write18{rm -f #2*.snm}
    \immediate\write18{rm -f #2*.dep}
    \immediate\write18{rm -f #2*.dpth} 
    %\immediate\write18{rm -f #2*.table} 
    %\immediate\write18{rm -f #2*.log} 


  } %To ensure group
}   


% DECORATION PATHS

% Style to hglight arrows with a white shadow
\tikzset{
    halo/.style={
        preaction={
            draw,
            white,
            line width=4pt,
            -
        },
        preaction={
            draw,
            white,
            ultra thick,
            shorten >=-2.5\pgflinewidth
        }
    }
}


% Sine function above a path (see http://tex.stackexchange.com/a/134516) this
% code gives some dimension too large
\tikzset{/pgf/decoration/.cd,
    number of sines/.initial=10,
    angle step/.initial=20,
}

\newdimen\tmpdimen
\pgfdeclaredecoration{complete sines}{initial}
{
    \state{initial}[
        width=+0pt,
        next state=move,
        persistent precomputation={
            \pgfmathparse{\pgfkeysvalueof{/pgf/decoration/angle step}}%
            \let\anglestep=\pgfmathresult%
            \let\currentangle=\pgfmathresult%
            \pgfmathsetlengthmacro{\pointsperanglestep}%
                {(\pgfdecoratedremainingdistance/\pgfkeysvalueof{/pgf/decoration/number of sines})/360*\anglestep}%
        }] {}
    \state{move}[width=+\pointsperanglestep, next state=draw]{
        \pgfpathmoveto{\pgfpointorigin}
    }
    \state{draw}[width=+\pointsperanglestep, switch if less than=1.25*\pointsperanglestep to final, % <- bit of a hack
        persistent postcomputation={
        \pgfmathparse{mod(\currentangle+\anglestep, 360)}%
        \let\currentangle=\pgfmathresult%
    }]{%
        \pgfmathsin{+\currentangle}%
        \tmpdimen=\pgfdecorationsegmentamplitude%
        \tmpdimen=\pgfmathresult\tmpdimen%
        \divide\tmpdimen by2\relax%
        \pgfpathlineto{\pgfqpoint{0pt}{\tmpdimen}}%
    }
    \state{final}{
        \ifdim\pgfdecoratedremainingdistance>0pt\relax
            \pgfpathlineto{\pgfpointdecoratedpathlast}
        \fi
   }
}



%ON BEAMER

% Daniel's code:
% http://tex.stackexchange.com/questions/55806/tikzpicture-in-beamer/55827#55827
  \tikzset{
    invisible/.style={opacity=0},
    visible on/.style={alt=#1{}{invisible}},
    alt/.code args={<#1>#2#3}{%
      \alt<#1>{\pgfkeysalso{#2}}{\pgfkeysalso{#3}} % \pgfkeysalso doesn't change the path
    },
  }



%% 3D coordinates rotation (http://tex.stackexchange.com/a/67588/28411)
% The matrix rotation is:
%$R=\begin{pmatrix}
%    \cos \alpha \cos \beta
%&   \textcolor{red}{\cos \alpha \sin \beta \sin \gamma - \sin \alpha \cos \gamma}
%&   \cos \alpha \sin \beta \cos \gamma + \sin \alpha \sin \gamma \\
%    \textcolor{red}{\sin \alpha \cos \beta}
%&   \sin \alpha \sin \beta \sin \gamma + \cos \alpha \cos \gamma
%&   \textcolor{red}{\sin \alpha \sin \beta \cos \gamma - \cos \alpha \sin \gamma} \\
%    - \sin \beta
%&   \textcolor{red}{\cos \beta \sin \gamma}
%&   \cos \beta \cos \gamma
%\end{pmatrix}\\p'=R\cdot p$

\newcommand{\savedx}{0}
\newcommand{\savedy}{0}
\newcommand{\savedz}{0}
 
\newcommand{\rotateRPY}[4][0/0/0]% point to be saved to \savedxyz, roll, pitch, yaw
{   \pgfmathsetmacro{\rollangle}{#2}
    \pgfmathsetmacro{\pitchangle}{#3}
    \pgfmathsetmacro{\yawangle}{#4}

    % to what vector is the x unit vector transformed, and which 2D vector is this?
    \pgfmathsetmacro{\newxx}{cos(\yawangle)*cos(\pitchangle)}% a
    \pgfmathsetmacro{\newxy}{sin(\yawangle)*cos(\pitchangle)}% d
    \pgfmathsetmacro{\newxz}{-sin(\pitchangle)}% g
    \path (\newxx,\newxy,\newxz);
    \pgfgetlastxy{\nxx}{\nxy};

    % to what vector is the y unit vector transformed, and which 2D vector is this?
    \pgfmathsetmacro{\newyx}{cos(\yawangle)*sin(\pitchangle)*sin(\rollangle)-sin(\yawangle)*cos(\rollangle)}% b
    \pgfmathsetmacro{\newyy}{sin(\yawangle)*sin(\pitchangle)*sin(\rollangle)+ cos(\yawangle)*cos(\rollangle)}% e
    \pgfmathsetmacro{\newyz}{cos(\pitchangle)*sin(\rollangle)}% h
    \path (\newyx,\newyy,\newyz);
    \pgfgetlastxy{\nyx}{\nyy};

    % to what vector is the z unit vector transformed, and which 2D vector is this?
    \pgfmathsetmacro{\newzx}{cos(\yawangle)*sin(\pitchangle)*cos(\rollangle)+ sin(\yawangle)*sin(\rollangle)}
    \pgfmathsetmacro{\newzy}{sin(\yawangle)*sin(\pitchangle)*cos(\rollangle)-cos(\yawangle)*sin(\rollangle)}
    \pgfmathsetmacro{\newzz}{cos(\pitchangle)*cos(\rollangle)}
    \path (\newzx,\newzy,\newzz);
    \pgfgetlastxy{\nzx}{\nzy};

    % transform the point given by #1
    \foreach \x/\y/\z in {#1}
    {   \pgfmathsetmacro{\transformedx}{\x*\newxx+\y*\newyx+\z*\newzx}
        \pgfmathsetmacro{\transformedy}{\x*\newxy+\y*\newyy+\z*\newzy}
        \pgfmathsetmacro{\transformedz}{\x*\newxz+\y*\newyz+\z*\newzz}
        \xdef\savedx{\transformedx}
        \xdef\savedy{\transformedy}
        \xdef\savedz{\transformedz}     
    }
}

\tikzset{RPY/.style={x={(\nxx,\nxy)},y={(\nyx,\nyy)},z={(\nzx,\nzy)}}}
\tikzset{RPYxzy/.style={x={(\nxx,\nxy)},z={(\nyx,\nyy)},y={(\nzx,\nzy)}}}





% HANDRAW

% http://tex.stackexchange.com/a/218483/28411
\pgfdeclaredecoration{penciline}{initial}{
  \state{initial}[width=+\pgfdecoratedinputsegmentremainingdistance,
    auto corner on length=1pt,
  ]{
    \ifthenelse
        {\pgfkeysvalueof{/tikz/penciline/jag ratio}=0} {
          \pgfpathcurveto%
              {% 1st control point
                \pgfpoint
                    {\pgfdecoratedinputsegmentremainingdistance/2}
                    {2*rnd*\pgfdecorationsegmentamplitude}
              }
              {%% 2nd control point
                \pgfpoint
                %% Make sure random number is always between origin and target points
                    {\pgfdecoratedinputsegmentremainingdistance/2}
                    {2*rnd*\pgfdecorationsegmentamplitude}
              }
              {% 2nd point (1st one is implicit)
                \pgfpointadd
                    {\pgfpointdecoratedinputsegmentlast}
                    {\pgfpoint{0*rand*1pt}{0*rand*1pt}}
              }          
        } {
          \pgfpathcurveto%
              {% 1st control point
                \pgfpoint
                    {\pgfdecoratedinputsegmentremainingdistance*rnd*1pt}
                    {\pgfkeysvalueof{/tikz/penciline/jag ratio}*
                      rand*\pgfdecorationsegmentamplitude}
              }
              {%% 2nd control point
                \pgfpoint
                %% Make sure random number is always between origin and target points
                    {(.5+0.25*rand)*\pgfdecoratedinputsegmentremainingdistance}
                    {\pgfkeysvalueof{/tikz/penciline/jag ratio}*
                      rand*\pgfdecorationsegmentamplitude}
              }
              {% 2nd point (1st one is implicit)
                \pgfpointadd
                    {\pgfpointdecoratedinputsegmentlast}
                    {\pgfpoint{rand*1pt}{rand*1pt}}
              }
        }
  }
  \state{final}{}
}

\tikzset{
  penciline/.code={\pgfqkeys{/tikz/penciline}{#1}},
  penciline={
    jag ratio/.initial=5,
    decoration/.initial = penciline,
  },
  penciline/.style = {
    decorate,
    %%decoration={\pgfkeysvalueof{/tikz/penciline/decoration}},
    penciline/.cd,
    #1,
    /tikz/.cd,
  },
  decorate,
  decoration={\pgfkeysvalueof{/tikz/penciline/decoration}},
}

%Use as \draw[penciline={jag ratio=6}]...


% The best is 
%http://tex.stackexchange.com/a/39299/28411
%pencildraw/.style={
%  black!75,
%  decorate,
%  decoration={random steps,segment length=2pt,amplitude=0.5pt}
%  } 

%%% http://tex.stackexchange.com/questions/39296/simulating-hand-drawn-lines: Alain Matthes
%\pgfdeclaredecoration{free hand}{start}
%{
%  \state{start}[width = +0pt,
%                next state=step,
%                persistent precomputation = \pgfdecoratepathhascornerstrue]{}
%  \state{step}[auto end on length    = 3pt,
%               auto corner on length = 3pt,               
%               width=+2pt]
%  {
%    \pgfpathlineto{
%      \pgfpointadd
%      {\pgfpoint{2pt}{0pt}}
%      {\pgfpoint{rand*0.15pt}{rand*0.15pt}}
%    }
%  }
%  \state{final}
%  {}
%}
% \tikzset{free hand/.style={
%    decorate,
%    decoration={free hand}
%    }
% } 
%\def\freedraw#1;{\draw[free hand] #1;}

% Line ends or Arrow styles

\tikzset{cuota/.style={
  postaction={decorate,
    decoration={markings,
      mark=at position 0 with  {
            \begin{scope}[xslant=0.2]
            \draw[line width=1pt,-] (0pt,-3pt) -- (0pt,3pt);
            \end{scope}
      },
      mark=at position 1 with  {
            \begin{scope}[xslant=0.2]
            \draw[line width=1pt,-] (0pt,-3pt) -- (0pt,3pt);
            \end{scope}
      }
    }
  }
}}
   




% To put a frame to an equation
\newcommand{\tikzframeab}[4]{%
  \tikz[remember picture,overlay] \path[#1]([shift={(-#2,#3)}] pic cs:a) rectangle ([shift={(#2,-#4)}] pic cs:b);
}
% Use:
% \begin{equation}
% \tikzmark{a} 2+2 \tikzmark{b}
% \end{equation}
% tikzframeab{draw}{2ex}{2.5ex}{1.5ex}
                                





% \usepackage[rows=10,cols=10]{beamersozi}
\tikzset{% 
  rec/.style={rectangle, draw=bg, rounded corners=1mm,fill=white,inner sep=3},
  recorte/.style={draw=bg,fill=white,draw, minimum height=2cm, minimum width=3cm,
         decorate, decoration={random steps,segment length=3,amplitude=2}},
}
 

\pgfdeclareradialshading{myshading}{\pgfpointorigin}{%
color(0mm)=(pgftransparent!0);%
color(5mm)=(pgftransparent!10);%
color(8mm)=(pgftransparent!50);%
color(15mm)=(pgftransparent!100)%
}
\pgfdeclarefading{myfading}{\pgfuseshading{myshading}}  
\tikzset{ partial ellipse/.style args={#1:#2:#3}{insert path={+ (#1:#3) arc (#1:#2:#3)}}}


\pgfdeclareradialshading{myshading2}{\pgfpointorigin}{%
color(0mm)=(pgftransparent!0);%
color(1mm)=(pgftransparent!50);%
color(5mm)=(pgftransparent!70);%
color(8mm)=(pgftransparent!80);%
color(15mm)=(pgftransparent!90);%
color(20mm)=(pgftransparent!100)%
}
\pgfdeclarefading{myfading2}{\pgfuseshading{myshading2}}  
\tikzset{ partial ellipse/.style args={#1:#2:#3}{insert path={+ (#1:#3) arc (#1:#2:#3)}}}

 
 
