\usepackage[thicklines]{cancel}
%\usepackage{cancel} %para simplificar en formulas
\usepackage{array} %matrices

\usepackage{amsmath} 
%para elaborar teoremas

% Environments Xmatrix with X={p, b, B, v, V} giving
% matrices with ( ), [ ], { }, | |, and || || delimiters
% There is also a matrix environment sans delimiters, and
% an array environment that can be used to obtain left
% alignment or other variations in the column specs. [ed.
% To produce a matrix with parenthesis around it, use:] 

\usepackage{amsfonts} %fuentes matematicas
%\mathnormal{…}  normal math font              most mathematical notation
%\mathrm{…}      normal font unitalicised      units of measurement, one word functions
%\mathit{…}      italicised font               difiere mucho de la normal en los numeros
%\mathbf{…}      bold font                     vectors
%\mathsf{…}      Sans-serif      
%\mathtt{…}      Monospace (fixed-width) font    
%\mathcal{…}     Calligraphy (uppercase only)  often used for sheaves/schemes and categories, used to denote cryptological concepts like an alphabet of definition (\mathcal{A}), message space (\mathcal{M}), ciphertext space (\mathcal{C}) and key space (\mathcal{K}); Kleene's \mathcal{O}; naming convention in description logic
%\mathfrak{…}**  Fraktur                       almost canonical font for Lie algebras, with superscript used to denote New Testament papyri, ideals in ring theory
%\mathbb{…}**    Blackboard bold               used to denote special sets (e.g. real numbers)
%\mathscr{…}*    Script                        an alternative font for categories and sheaves
%*requires amsfonts or amssymb package
%**require mathrsfs package

\usepackage{amssymb} %simbolos matematicos


\usepackage{fixmath} %simbolos matematicos arreglados??

%\newtheorem{theorem}{Theorem}
%\usepackage{dsfont}   %mathds para letras de conjuntos como R Z Q y N
\usepackage{latexsym} %Some thinks like arrows
 
% Shortcuts
% ---------
\newcommand{\ds}{\displaystyle}

% Typesetting equations for physics made simpler
\usepackage{physics2}

\usephysicsmodule{ab}
% \ab is an `automatic delimiter`.
% 	works with (), [], \{\}, <>, || and \|\|.
%   takes an optional argument that can be \big, \Big, etc
% \ab* prevent the automatic resize of \ab.
% \Xab{} with X={p, b, B, a, v, V} is \ab using {} syntax
%   note the X follows amsmath convention
%   i.e. \pab{a} is equal to \ab(a)

\usephysicsmodule{ab.braket}
% % NOTE: Conflicts with braket
% % \bra, \ket, \braket and \ketbra

% \usephysicsmodule{braket}
% % NOTE: Conflicts with ab.braket
% % \bra, \ket, \braket and \ketbra
 
\usephysicsmodule{xmat}
% Matrices with formatted entries:
%   \xmat[⟨options⟩]{⟨entry⟩}{⟨rows⟩}{⟨cols⟩}
% Options like showleft=X or showtop=Y serves to add dots afer X columns or
% Y rows.
% To add delimters follows amsmath convention:
%   \pxmat, \bxmat, \Bxmat, \vxmat and \Vxmat 
%   ( ), [ ], { }, | |, and || ||
                    
% \usephysicsmodule{ab.legacy} % Not recommended
% % \abs, \norm, \eval, (\peval, \beval), \order
% %   they use same syntax than \ab

% \usephysicsmodule{nabla.legacy} % Not recommended
% % \grad, \div, \curl

% \usephysicsmodule{op.legacy} % Not recommended
% \asin, \acos, \atan, ...

% Bold math
% ---------

% TODO: ver physics2 module bm-um.legacy

%BUG:
%\newcommand\hmmax{0} % default 3
\newcommand\bmmax{0} % default 4
%Notice use of \newcommand (not \renewcommand), and that this is
%inserted BEFORE the bm package.
%
%It's not actually the AMS packages that are involved in the solution
%but they were part of the problem, as it were. The bm package is of
%course the `Bold Math' package -- but because of its purpose, to
%expand the number of fonts available, it also has the necessary
%mechanism for coping with the message:
%
%          Too many math alphabets used in version normal.
%
% You may need to adjust the above value, see bm docs section 2, p.2.

\usepackage{bm}
% Replace amsmath \boldsymbol and provide \bm command.
% 1. keeps the italic correction, i.e. \bm{T}_1^2 will look better than
% \boldsymbol{T}_1^2. To compare be careful that bm redefines \boldsymbol. 
% 2. avoid the spacing disruption of \boldsymbol.
% NOTE: sometimes needs extra braces as \bm{{\dots}} to avoid error with amsmath. 
% NOTE: math nesting works better with \boldsymbol. For example,
%   \mathrm{g\boldsymbol{g}} work and \mathrm{g\bm{g}} does not.
        
% \usephysicsmodule{bm.legacy} % Not recommended
% The \bm command from bm package uses \mathversion to
% support its func- tion, but there are few OpenType math
% fonts who released with a bold version.  The bm-um.legacy
% module provides a \bm command too, but this \bm can only
% take one math character or a series of math characters
% sharing the same category code as its argument. If the
% argument was Latin letters or Greek letters, \bm would
% switch to the bold italic glyphs corresponding to them
% (if there exists bold italic glyphs); else \bm would
% switch to the bold upright glyphs.
    

 
% REVIZAR
%--------

\newlength{\unit} % La unidad fundamental (Para TikZ y lo que se necesite)

%delimitadores
%-------------

%\usepackage{nath} %change the effect of left and right
%\delimgrowth=1

%maximum space not covered by a delimiter
\delimitershortfall=0pt %from amsmath, default 5pt

%ratio for variable delimiters, times 1000 
%\delimiterfactor=901 %from amsmath, default 901

% the last two command can be locally defined. Here the macro:
\newcommand{\forcegrow}{
 \setlength{\delimitershortfall}{-1pt}
}

% For brake line delimiters see
% http://tex.stackexchange.com/a/124135/28411

% \usepackage{mathtools}
% % NOTE: Consider using physics2
% \DeclarePairedDelimiter\ds{\displaystyle{#1}}%

% In case of ccfonts uncomment this
%\let\underbrace\LaTeXunderbrace
%\let\overbrace\LaTeXoverbrace
% In super or under scripts it is needed to add the \scriptstyle manually

\newcommand{\soverbrace}[2][]{%
{\everymath{\scriptstyle}%
 \overbrace{\scriptstyle#2}^{#1}}%
}
 
\newcommand{\sunderbrace}[2][]{%
{\everymath{\scriptstyle}%
 \underbrace{\scriptstyle#2}_{#1}}%
}

% MATEMATICA
% ----------

% Math abreviations
%Estan ya estan en alguno de los paquetes
%BOLD {bb}{\mathbb}{ABCDEFGHIJKLMNOPQRSTUVWXYZ}
%{bf}{\mathbf}{ABCDEFGHIJKLMNOPQRSTUVWXYZabcdefghijklmnopqrstuvwxyz}
%{bit}{\boldsymbol}{ABCDEFGHIJKLMNOPQRSTUVWXYZabcdefghijklmnopqrstuvwxyz}
%{cal}{\mathcal}{ABCDEFGHIJKLMNOPQRSTUVWXYZ}
%{frak}{\mathfrak}{ABCDEFGHIJKLMNOPQRSTUVWXYZabcdefghijklmnopqrstuvwxyz}
%{rm}{\mathrm}{ABCDEFGHIJKLMNOPQRSTUVWXYZabcdefghijklmnopqrstuvwxyz}
%{scr}{\mathscr}{ABCDEFGHIJKLMNOPQRSTUVWXYZ}
%{sf}{\mathsf}{ABCDEFGHIJKLMNOPQRSTUVWXYZabcdefghijklmnopqrstuvwxyz}
 


\newcommand{\Evib}{\ensuremath{\mathrm{E_{vib}}}}
\newcommand{\Erot}{\ensuremath{\mathrm{E_{rot}}}}
\newcommand{\Ecin}{\ensuremath{\mathrm{E_{cin}}}}
\newcommand{\Epot}{\ensuremath{\mathrm{E_{pot}}}}

%\newcommand{\bwrb}{\ensuremath{\mathbf{w}^{cr}}}
  
\newcommand{\cdf}{\ensuremath{\text{cdf}}}
\newcommand{\cdfsn}{\ensuremath{\text{cdf}_{\text{sn}}}}
\newcommand{\erf}{\ensuremath{\mathrm{erf}}}
\newcommand{\erfc}{\ensuremath{\mathrm{erfc}}}

%Ecuations
\newcommand{\dsum}{\displaystyle \sum}

%Limite
\newcommand{\lm}[2]{\lim_{#1\rightarrow #2}} 

%Separadores
\newcommand{\sep}{\ifhmode\par\fi\vskip6pt\hrule\vskip20pt}

% Derivadas
\usepackage{derivative}
% \Xdv*[⟨keyval list⟩]{⟨function⟩}/!{⟨variables⟩}_{⟨point1⟩}^{⟨point2 \pdv ⟩}
% with X={p,o,m,f,a,j} for partial, ordinary, material,
% functional, average and jacobian derivatives. Also:
%   * derivative as an operator (i.e. function outside the fraction)
%   / using slash instead of frac (i.e. a/b) 
%   ! subindex to indicate independent variable 
%   /! like subindex, but with common letters
% \Xdif for diferentials (e.g. try \odif{x,y})

% derivative can be used also with:  
% \usepackage{fixdif}


% Functions in <physics.sty>

% \def\drinline#1#2{\partial #1 \over \partial #2}  % primera derivada parcial en una linea
% \def\Drinline#1#2{d #1/d #2}                      % primera total en una linea
% \def\dr#1#2{\frac{\partial #1}{\partial #2}}      % primera derivada parcial
% \def\ddr#1#2{\frac{\partial^2 #1}{\partial #2^2}} % segunda derivada parcial
% \newcommand[3][{}]{\Dr}{\frac{d^{#1}#2}{d#3^{#1}}}                    % primera derivada total
% \def\DDr#1#2{\frac{d^2 #1}{d #2^2}}               % segunda derivada total
% \def\DDDr#1#2{\frac{d^3 #1}{d #2^3}}               % segunda derivada total
% \def\drz#1#2#3{\left(\frac{\partial #1}{\partial #2}\right)_{#3}}
%\newcommand{\bmdot}[1]{\ensuremath{\bm{{\dot{#1}}}}}
%\newcommand{\bmddot}[1]{\ensuremath{\bm{{\ddot{#1}}}}}

% Esto es para corregir desplazamientos en derivadas de punto de orden superior
\makeatletter
\renewcommand{\dddot}[1]{%
  {\mathop{\kern\z@
  {\mathop{\kern\z@
  #1
  }\limits^{\vbox to-1.4\ex@{\kern-\tw@\ex@
     \hbox{\normalfont ..}\vss}}} 
  }\limits^{\vbox to-2.0\ex@{\kern-\tw@\ex@
     \hbox{\normalfont .}\vss}}} 
}
%\newcommand{\dddotvec}[1]{%
%  {\mathop{\kern\z@
%  {\mathop{\kern\z@
%  \vec{#1}
%  }\limits^{\vbox to-1.4\ex@{\kern-\tw@\ex@
%     \hbox{\dot}\vss}}} 
%  }\limits^{\vbox to-2.0\ex@{\kern-\tw@\ex@
%     \hbox{\bf  .}\vss}}} 
%}
%\renewcommand{\ddddot}[1]{%
%  {\mathop{\kern\z@
%  {\mathop{\kern\z@
%  #1
%  }\limits^{\vbox to-1.4\ex@{\kern-\tw@\ex@
%     \hbox{\normalfont ..}\vss}}} 
%  }\limits^{\vbox to-1.4\ex@{\kern-\tw@\ex@
%     \hbox{\normalfont ..}\vss}}} 
%}
%\newcommand{\dddddot}[1]{%
%  {\mathop{\kern\z@
%  {\mathop{\kern\z@
%  #1
%  }\limits^{\vbox to-1.4\ex@{\kern-\tw@\ex@
%     \hbox{\normalfont ...}\vss}}} 
%  }\limits^{\vbox to-1.4\ex@{\kern-\tw@\ex@
%     \hbox{\normalfont ..}\vss}}}
%}
\makeatother 
\renewcommand{\ddddot}[2]{\ensuremath{\overset{(#1)}{#2}}}

\newcommand{\dotvec}[1]{\ensuremath{\bm{{\dot{#1}}}}} %Combiene usar este por si se desea cambiar la notacion a flechas
\newcommand{\ddotvec}[1]{\ensuremath{\bm{{\ddot{#1}}}}}
\newcommand{\dddotvec}[1]{\ensuremath{\bm{{\dot{\ddot{#1}}}}}}
\newcommand{\ddddotvec}[2]{\ensuremath{{\overset{(#1)}{\vec{#2}}}}}

\newcommand{\ver}[1]{\ensuremath{\bm{{\hat{#1}}}}} %Combiene usar este por si se desea cambiar la notacion a flechas
                
%\def\dotproduct#1#2{\left\langle #1\cdot#2 \right\rangle}
\def\inner#1#2{\left\langle #1\middle|#2 \right\rangle}
\def\dotproduct#1#2{\left( #1\cdot#2 \right)}
\def\ddotproduct#1#2{#1\cdot#2}

%	Integrales
%	\def\oiint{\int\hspace{-2ex}\int\hspace{-3ex}\bigcirc~}
%	\def\iint{\int\hspace{-1.5ex}\int}
%	\def\iiint{\int\hspace{-1.5ex}\int\hspace{-1.5ex}\int}


% Algebra
% =======

% Physics2 suggestion to replace old physics support
\makeatletter
\newcommand\vb{\@ifstar\boldsymbol\mathbf}
\newcommand\va[1]{\@ifstar{\vec{#1}}{\vec{\mathrm{#1}}}}
\newcommand\vu[1]{%
\@ifstar{\hat{\boldsymbol{#1}}}{\hat{\mathbf{#1}}}}
\makeatother
% \vb means bold vector
% \va means arrow vector
% \vu means bold versor (unit vector)
% However, the method above may not work well with unicode-math because there
% are so many OpenType math fonts without a bold version. When using
% unicode-math, it’s recommended to use \symbf and \symbfit for a separate
% vector. This works for both Greek and Latin letters.

% Different bold methods (deprecated)
%\renewcommand{\vec}[1]{\mbox{\boldmath$#1$\unboldmath}} 
%\renewcommand{\vec}[1]{\boldsymbol{#1}} % amsmath bold vectors.
%\renewcommand{\vec}[1]{\ensuremath{\bm{{#1}}} % Uncomment for BOLD vectors.
                                     
 
% Puntos seguidos (\dots)
% se puede usar \dots, \ldots, \vdots, \cdots, \ddots, pero
% con el paquete AMS cuando el uso es matematicamente conocido
% se puede aprovechar estos comandos cuya letra recuerda donde
% matematicamente se usa
%  \dotsc is for dots with commas    (a1,...,an)
%  \dotsb is for dots with binary operators
%  \dotsm is for multiplication dots
%  \dotsi is for dots with integrals
%  \dotso is for other dots


% Complejos
% ---------

% Change default symbols to amsmath operators 
\renewcommand{\Re}{\operatorname{Re}}
\renewcommand{\Im}{\operatorname{Im}}


% Conjuntos
% ---------

\newcommand{\R}{\mathbb{R}}% negrita
\newcommand{\N}{\mathbb{N}}% 
\newcommand{\Z}{\mathbb{Z}}% 
\newcommand{\Q}{\mathbb{Q}}

%\newcommand{\R}{\mathds{R}}% vacia
%\newcommand{\N}{\mathds{N}}
%\newcommand{\Z}{\mathds{Z}}
%\newcommand{\Q}{\mathds{Q}}

%\newcommand{\set}{\mathds{Q}}

%\newcommand{\en}[1]{\ensuremath{ \in \mathbb{#1}}}
\newcommand{\cjSol}[2]{\ensuremath{\text{Sol}=\lbrace#1\en{R}/#2\rbrace}}
% La ``y'' en conjuntos es \wedge
% La ``o'' en conjuntos es \vee
% No pertenece es \not \in
 


% Matrices
% --------

%\newcommand{\arr}[1]{\bm{{#1}}} % Uncomment for BOLD vectors.

% Operadores
% ----------

% Algunos fueron sacados de:
% Physics Formulary by ir. J.C.A. Wevers <johanw@xs4all.nl>

\newcommand{\op}[1]{\ensuremath{\hat{\mathrm{#1}}}} %con sombrerito
%\newcommand{\op}[1]{\ensuremath{\bm{{#1}}}}         %con negrita

%\newfont{\sfd}{cmssdc10 scaled\magstep0}
%\newfont{\cmu}{cmu10 scaled\magstep0}

%\def\jwline#1#2#3#4{%
%\put(#1,#2){\special{em:moveto}}%
%\put(#3,#4){\special{em:lineto}}}
%\def\newpic#1{}

%\def\npar{\par\medskip}
%\def\LL{{\cal L}}
%\def\RR{I\hspace{-1mm}R}
%\def\TT{\mbox{\sfd T}}
%\def\DD{\mbox{\sfd D}}
%\def\half{\mbox{$\frac{1}{2}$}}
%\def\kwart{\mbox{$\frac{1}{4}$}}
%\def\dd{d\hspace{-1ex}\rule[1.25ex]{2mm}{0.4pt}}
%\def\lrarrow{~\lower.2ex\hbox{$\rightarrow$}\kern-2.4ex\raise.7ex\hbox{$\leftarrow$}~}
%\def\rlarrow{~\lower.2ex\hbox{$\leftarrow$}\kern-2.3ex\raise.7ex\hbox{$\rightarrow$}~}

% Some aliases
% -----------

% Media 
\newcommand{\avg}[1]{\overline{#1}} 

       
% Problema
% --------

\newcounter{problema}

\newenvironment{problemasanwich}
{
  \flushleft
  \stepcounter{problema}%
  \rule{0.43\linewidth}{0.08ex}
  \textbf{\noindent{Problema \theproblema}}
  \rule{0.43\linewidth}{0.08ex}\\[0.5cm]
}{
\noindent
\vspace{-1ex}\rule{\linewidth}{0.08ex}\\[0.5cm]
}

\newenvironment{problemasep}
{
  \flushleft
  \stepcounter{problema}%
  \rule{0.43\linewidth}{0.08ex}
  \textbf{\noindent{Problema \theproblema}}
  \rule{0.43\linewidth}{0.08ex}\\[0.5cm]
}{
}
 
% Comentario
\def\comment#1{{\scriptsize\bf\color{red}#1}}

%Slanted frctions ?????
% \[ \left. {}^{3} \right/ \! \left. _{4} \right. \]

%Llaves de varios renglones
\newenvironment{rcase}
  {\left.\begin{aligned}}
  {\end{aligned}\right\rbrace}

\newenvironment{lcase}
  {\left\lbrace \begin{aligned}}
  {\end{aligned}\right.}

% Simbolos   
% --------
% Para evitar romper el higliting del editor cuando se
% necesita usar estos simbolos desapareados. Lo pongo aca
% abajo, para mantener el higlighting en este archivo.
\def\asterisco{*}
\def\virguilla{~}
\def\comillas{"}
\def\comilla{'}
\newcommand{\Amstr}{\textup \AA }
\def\pesos{$} 
